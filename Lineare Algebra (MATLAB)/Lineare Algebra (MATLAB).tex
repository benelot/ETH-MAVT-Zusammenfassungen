% HEADER (fold)
\documentclass[10pt]{article}

\author{Philipe Fatio}
\title{Lineare Algebra \\ Zusammenfassung}
\date{\today}

\usepackage[
	a4paper,
	landscape,
	left=1.25cm,
	right=1.25cm,
	top=0.75cm,
	bottom=0.75cm,
	includeheadfoot
]{geometry}

\usepackage{multicol}
\setlength{\columnseprule}{.5pt}
\setlength{\columnsep}{1cm}
\setlength{\parindent}{0cm}

% Wir schreiben hier auf Deutsch
\usepackage[german]{babel}
\usepackage{pdfpages}

% Setup for fullpage use
%\usepackage[margin=2.5cm,includeheadfoot]{geometry}

% Enable syncing with PDF view
%\usepackage{pdfsync}

% Use utf-8 encoding for foreign characters
\usepackage[utf8]{inputenc}

% Nice headers
\usepackage{fancyhdr}
\fancyhf{}
\cfoot{-- \emph{\thepage} --}
\rhead{\textit{Philipe Fatio} -- \today}
\lhead{\textbf{MATLAB Zusammenfassung}}
\setlength{\headheight}{14pt}
\pagestyle{fancy}


\usepackage{titlesec}
\titleformat*{\section}{\large \bf}
\titleformat*{\subsection}{\normalsize \bf}
\titleformat*{\subsubsection}{\normalsize \bf}
\titleformat*{\paragraph}{\footnotesize \bf}
\titleformat*{\subparagraph}{\footnotesize \bf}

% \makeatletter
% \renewcommand\paragraph{\@startsection*{paragraph}{4}{\z@}%
%   {-3.25ex\@plus -1ex \@minus -.2ex}%
%   {1.5ex \@plus .2ex}%
%   {\normalfont\normalsize\bfseries}}
% \makeatother

% Use Mathematical Equations
\usepackage{amsmath}
\usepackage{bbm}

% Set the paragraph style
%\setlength{\parindent}{10pt}
\setlength{\parskip}{.5ex plus 0.25ex minus 0.1ex}

% HEADER (end)
\begin{document}
	\begin{multicols*}{3}
		\section*{Grundlegendes} % (fold)
			Ausgabe einer Zeile wird mit abschliessendem Punktstrich unterdrückt.
			
			Wurzeln werden per \verb+sqrt+ eingegeben. Die Wurzel von 2 wäre \verb+sqrt(2)+.
			
			\verb+clear A+ löscht die Variable \verb+A+.
			
			Anhand \verb+format long+ wird eine höhere Präzision bei den Fliesskommazahlen erreicht.
		% section: Grundlegendes (end)
		\subsection*{Eingabe von Matrizen} % (fold)
			Matrizen werden Zeilenweise eingegeben. Falls eine Matrix aus mehreren Zeilen besteht, werden die einzelnen Zeilen per Punktstrich getrennt.
			
			\begin{verbatim}
				A = [-6 -1  2;    x = [ 0;
				      4  3 -5;         -2;
				      1 -2  3]          4]
			\end{verbatim}
			\subsubsection*{Einheitsmatrix} % (fold)
				Eine Einheitsmatrix kann anhand der Funktion \verb+eye+ erzeugt werden, welche als einzigen Parameter die Dimension der Matrix hat.
				\begin{verbatim}
					I_2 = eye(2)
				\end{verbatim}
			% subsubsection: Einheitsmatrix (end)
			\subsubsection*{Teilmatrix} % (fold)
				Einen Bereich einer Matrix erhält man mit der Syntax \verb+A(Startzeile:Endzeile,Startspalte:Endspalte)+.
				\begin{verbatim}
					A(2:3,1:2) -> [4  3;
					               1 -2]
				\end{verbatim}
			% subsubsection: Teilmatrix (end)
			\subsubsection*{Matrizen generieren} % (fold)
				Eine Matrix mit einer linearen Folge kann per \verb+[Startwert:Schrittgrösse:Maximalwert]+  definiert werden.
				\begin{verbatim}
					y = [1:1:4] -> [1 2 3 4]
				\end{verbatim}
			% subsubsection: Matrizen generieren (end)
		% subsection: Eingabe von Matrizen (end)
		\section*{Operationen} % (fold)
			\subsection*{Transponieren} % (fold)
				Um eine Matrix zu transponieren wird an die die Matrix ein \verb+'+ angehängt.
				Somit kann man auch einfacher Vektoren schreiben:
				\begin{verbatim}
					x = [0 -2 4]'
				\end{verbatim}
			% subsection: Transponieren (end)
			\subsection*{Inverse} % (fold)
				Die Inverse einer Matrix \verb+A+ kann per \verb+inv(A)+ berechnet werden.
			% subsection: Inverse (end)
			\subsection*{Lösen von Gleichungssystemen} % (fold)
				Durch Linksdivision kann ein Gleichungssystem der Form \verb+Ax = b+ gelöst werden.
				\begin{verbatim}
					x = A\b
				\end{verbatim}
			% subsection: Lösen von Gleichungssystemen (end)
			\subsection*{Lösen von Fehlergleichungen} % (fold)
				Mehrere Methoden:
				\begin{itemize}
					\item durch Linksdivision: \verb+x = A\c+
					\item \verb+x = regress(c, A)+
					\item \verb+x = linsolve(A'*A, A'*c)+
					\item mit QR--Zerlegung
				\end{itemize}
			% subsection: Lösen von Fehlergleichungen (end)
			\subsection*{Determinanten} % (fold)
				Determinante der Matrix \verb+A+:
				\begin{verbatim}
					det(A)
				\end{verbatim}
			% subsection: Determinanten (end)
			\subsection*{LR--Zerlegung} % (fold)
				LR-Zerlegung einer Matrix \verb+A+:
				\begin{verbatim}
					[L,R,P] = lu(A)
				\end{verbatim}
				Die Lösung des Gl.sys. \verb+Ax = b+ ist dann
				\begin{verbatim}
					c = L\(P*b);
					x = R\c
				\end{verbatim}
			% subsection: LR--Zerlegung (end)
			\subsection*{QR--Zerlegung} % (fold)
				QR-Zerlegung einer Matrix \verb+A+:
				\begin{verbatim}
					[Q,R] = qr(A);
					d = Q'*c;
					x = R(1:n,1:n)\d(1:n)
				\end{verbatim}
				wobei \verb+n+ die Anzahl Spalten der Matrix \verb+A+ ist.
			% subsection: QR--Zerlegung (end)
			\subsection*{Eigenwerte} % (fold)
				Eigenwerte und -vektoren werden wie folgt bestimmt. Dabei sind in der Diagonalen von \verb+D+ die Eigenwerte und in den Spalten von \verb+V+ die Eigenvektoren gespeichert:
				\begin{verbatim}
					[V,D] = eig(A)
				\end{verbatim}
			% subsection: Eigenwerte (end)
			\subsection*{Norm} % (fold)
				Die Norm einer Matrix kann anhand der Funktion \verb+norm+ berechnet werden. Als zweiter Parameter wird angegeben, welche Norm berechnet werden soll. Falls dieser nicht angegeben wird, wird die 2-Norm berechnet. Möglich wäre auch \verb+1+ oder \verb+inf+.
				\begin{verbatim}
					norm(A)
					norm(A,inf)
				\end{verbatim}
			% subsection: Norm (end)
			\subsection*{Rang} % (fold)
				\begin{verbatim}
					rank(A)
				\end{verbatim}
			% subsection: Rang (end)
			\subsection*{Singulärwerte} % (fold)
				Singulärwerte sind in der Diagonalen von \verb+S+ gespeichert:
				\begin{verbatim}
					[U,S,V] = svd(A)
				\end{verbatim}
			% subsection: Singulärwerte (end)
		% section: Operationen (end)
	\end{multicols*}
\end{document}
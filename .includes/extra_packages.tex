%% See the TeXed file for more explanations

\usepackage{babel}

\usepackage[utf8]{inputenc}
\usepackage[sc]{mathpazo}
\usepackage[thmmarks]{ntheorem}
\usepackage[nice]{nicefrac}
\usepackage[load={prefixed,prefix,abbr,named,synchem,addn},per=frac,fraction=frac]{siunitx}
\usepackage{tikz}
\usepackage{lmodern}
\usepackage[T1]{fontenc}

%% Math related packages

%% The AMS-LaTeX extensions for mathematical typesetting.
\usepackage{amsmath,amssymb,amsfonts,mathrsfs}

\renewcommand{\arraystretch}{0.8}
% import content
\newcommand{\content}[1]{
	\input{content/#1.tex}
}
% Macro for units
\newcommand{\sunit}[1]{\ensuremath{ \quad \left[ \si{#1} \right] } }
\newcommand{\nicesunit}[1]{\ensuremath{ \quad \left[ \si[fraction=nicefrac]{#1} \right] } }
\newcommand{\unit}[1]{\ensuremath{ \left[ \si{#1} \right] } }
\newcommand{\niceunit}[1]{\ensuremath{ \left[ \si[fraction=nicefrac]{#1} \right] } }

%% [OPT] Multi-rowed cells in tabulars
%\usepackage{multirow}

%% [REC] Intelligent cross reference package. This allows for nice
%% combined references that include the reference and a hint to where
%% to look for it.
\usepackage{varioref}
% \labelformat{equation}{(#1)} % keine gute idee

%% [OPT] Easily changeable quotes with \enquote{Text}
% \usepackage[german=swiss]{csquotes}

%% [REC] Format dates and time depending on locale
% \usepackage{datetime}

%% [OPT] Provides a \cancel{} command to stroke through mathematics.
\usepackage{cancel}

%% [ADV] This allows for additional typesetting tools in
%% mathmode. \mathclap{} will allow you to take away the width of a
%% mathematical statement seen by LaTeX.
\usepackage{mathtools}

%% [ADV] Conditional commands
%% (note that this is included by macrosetup.tex anyway)
%\usepackage{ifthen}

%% [OPT] Manual large braces or other delimiters.
%\usepackage{bigdelim, bigstrut}

%% [REC] Alternate vector arrows. Use the command \vv{} to get scaled
%% vector arrows.
\usepackage[h]{esvect}

%% [NEED] Some extensions to tabulars and array environments.
\usepackage{array}

%% [OPT] Provides \unit[1]{N} as a means to facilitate unit
%% typesetting as well as \nicefrac{}{} command that prints a fraction
%% in text-height. Very useful for fractions inside matrices.
%\usepackage{units}

%% [NEED] Allows to rotate elements
\usepackage{rotating}

%% [OPT] LaTeX epic/eepic graphics format support.
%\usepackage{epic,eepic}

%% [OPT] Postscript support via pstricks graphics package. Very
%% diverse applications.
%\usepackage{pstricks,pst-all}

%% [?] This seems to allow us to define some additional counters.
%\usepackage{etex}

%% [ADV] XY-Pic to typeset some matrix-style graphics
%\usepackage[all]{xy}

\usepackage{booktabs}
% \usepackage{tabulary}

% \usepackage[caption=false,format=hang]{subfig}

% \usepackage[pdftex,bookmarksopen]{hyperref}

\makeatletter

\def\@pdfborder{0 0 0}%
\let\@pdfborderstyle\@empty

\makeatother

\usepackage{longtable}

\usepackage{etex}

\usepackage{fancybox}
\setlength\shadowsize{1pt}
\usepackage{empheq}

\usepackage[setpagesize=false]{hyperref}

\usepackage{multirow}
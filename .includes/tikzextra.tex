\usetikzlibrary{decorations}
\usetikzlibrary{arrows}

\usetikzlibrary{%
    decorations.pathreplacing,%
    decorations.pathmorphing%
}

\usetikzlibrary{calc}

\makeatletter
\pgfarrowsdeclare{smo}{smo}
{
  \pgfarrowsleftextend{+-.5\pgflinewidth}
  \pgfutil@tempdima=0.2pt%
  \advance\pgfutil@tempdima by.2\pgflinewidth%
  \pgfutil@tempdimb=9\pgfutil@tempdima\advance\pgfutil@tempdimb by.5\pgflinewidth
  \pgfarrowsrightextend{+\pgfutil@tempdimb}
}
{
  \pgfutil@tempdima=0.2pt%
  \advance\pgfutil@tempdima by.2\pgflinewidth%
  \pgfsetdash{}{+0pt}
  \pgfpathcircle{\pgfqpoint{4.5\pgfutil@tempdima}{0bp}}{4.5\pgfutil@tempdima}
  \pgfusepathqstroke
}

\makeatother

\tikzset{interface/.style={
        % The border decoration is a path replacing decorator. 
        % For the interface style we want to draw the original path.
        % The postaction option is therefore used to ensure that the
        % border decoration is drawn *after* the original path.
        postaction={draw,decorate,decoration={border,angle=-45,
                    amplitude=0.3cm,segment length=2mm}}},
	feder/.style={
		decorate,decoration={coil,post length=.15cm, pre length=.15cm},segment length=4pt
	}
}

\newcommand{\kreisel}[3]{
\def\ellAdj{.43}
\path[draw,fill=gray!20] (#2/2/#3,0) ellipse (\ellAdj*#1*#3 and #1) ;
\path[draw,shade,bottom color=gray!40,top color=gray!30,middle color=gray!5] (#2/2/#3, #1) -- ++ (-#2/#3,0) arc (90:270:\ellAdj*#1*#3 and #1) -- ++ (#2/#3,0) arc (270:90:\ellAdj*#1*#3 and #1);
}

\newcommand{\measurement}[4][]{%
\def\ruleInset{3pt}

\draw #2 -- ++ #4 coordinate(tmpa) ++ ($(0,0)!1!90:#4$) coordinate(tmpb);
\draw #3 -- ($(tmpa)!#3!(tmpb)$) coordinate(tmpc);

\draw[<->] (tmpa) ++ ($(0,0)!-\ruleInset!#4$) -- ($(tmpc) + (0,0)!-\ruleInset!#4$) node[midway,sloped,below]{#1};
}

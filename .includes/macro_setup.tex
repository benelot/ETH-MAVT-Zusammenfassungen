%% Custom commands =============================================================

%% Depends on this
\usepackage{ifthen}
\usepackage{rotating}
\usepackage{xy}

%% Differential d. This sets it as proper operator in roman type. With correct
%% spacing. ISO standards for mathematical typesetting says it should be printed
%% like this.
\newcommand{\diff}[1]{\operatorname{d}\ifthenelse{\equal{#1}{}}{\,}{\!#1}}

% Einheitsmatrix
\newcommand{\I}{\mathbb{I}}

%% Symbols for euler number and imaginary unit
\providecommand*{\eu}%
{\ensuremath{\mathrm{e}}}
% The imaginary unit
\providecommand*{\iu}%
{\ensuremath{\mathrm{i}}} % i can be replaced with j on preference.

%% Ortsvektor mit römischen buchstaben
\newcommand{\pVec}[1]{\vv{r_{#1}}}

%% Uncomment below what style you prefer for printing differential operators
%\DeclareMathOperator{\grad}{grad}
%\DeclareMathOperator{\rot}{rot}
%\DeclareMathOperator{\Div}{div}
\DeclareMathOperator{\grad}{\nabla\!}
\DeclareMathOperator{\rot}{\nabla\times}
\DeclareMathOperator{\Div}{\nabla\cdot}

% complex and real operators
\renewcommand{\Im}{\mathrm{Im}}
\renewcommand{\Re}{\mathrm{Re}}

%% Additional mathematical operators
\DeclareMathOperator{\tr}{tr}
\DeclareMathOperator{\id}{Id}
\DeclareMathOperator{\Kern}{Kern}
\DeclareMathOperator{\diag}{diag}
\DeclareMathOperator{\arccot}{arccot}
\DeclareMathOperator{\arsinh}{arsinh}
\DeclareMathOperator{\arcosh}{arcosh}
\DeclareMathOperator{\artanh}{artanh}
\DeclareMathOperator{\const}{const}
\DeclareMathOperator{\erf}{erf}
\DeclareMathOperator{\erfc}{erfc}
\newcommand{\transp}{\mathsf T}

\newcommand{\bVec}[1]{\mathbold{#1}}

%% German variants
%\DeclareMathOperator{\Kern}{Kern}
%\DeclareMathOperator{\Bild}{Bild}
%\DeclareMathOperator{\Grad}{Grad}
%% English variants
%% \ker is provided by LaTeX
\DeclareMathOperator{\im}{im}
%% \grad is provided by LaTeX

%% Special characters for number sets, e.g. real or complex numbers.
\newcommand{\C}{\mathbb{C}}
\newcommand{\K}{\mathbb{K}}
\newcommand{\N}{\mathbb{N}}
\newcommand{\Q}{\mathbb{Q}}
\newcommand{\R}{\mathbb{R}}
\newcommand{\Z}{\mathbb{Z}}
\newcommand{\X}{\mathbb{X}}

%% Fixed size delimiter examples
\newcommand{\floor}[1]{\lfloor #1 \rfloor}
\newcommand{\ceil}[1]{\lceil #1 \rceil}
\newcommand{\seq}[1]{\langle #1 \rangle}
\newcommand{\set}[1]{\{ #1 \}}
\newcommand{\abs}[1]{\lvert #1 \rvert}
\newcommand{\norm}[1]{\lVert #1 \rVert}
\newcommand{\indic}[1]{\bigl[#1\bigr]}

%% Scaling delimiter examples
\newcommand{\Floor}[1]{\left\lfloor #1 \right\rfloor}
\newcommand{\Ceil}[1]{\left\lceil #1 \right\rceil}
\newcommand{\Seq}[1]{\left\langle #1 \right\rangle}
\newcommand{\Set}[1]{\left\{ #1 \right\}}
\newcommand{\Abs}[1]{\left\lvert #1 \right\rvert}
\newcommand{\Norm}[1]{\left\lVert #1 \right\rVert}

%% Absolute and partial derrivate fractions
\newcommand{\Diff}[2]{\displaystyle\frac{\diff{#1}}{\diff{#2}}}
\newcommand{\Part}[2]{\displaystyle\frac{\partial #1}{\partial #2}}

%% Set an index and print it to the current position at the same time
\newcommand{\Index}[1]{\emph{#1}\index{#1}}

%% Displaystyle math for inline math mode
\newcommand{\ds}{\displaystyle}

%% Easy to use alias for the default matrices with round braces
\newcommand{\Mx}[1]{\ensuremath{\begin{bmatrix}#1\end{bmatrix}}}

%% Include a lecture from the lectures/ folder by date.
\newcommand{\Include}[4][\prefix]%
{\ifthenelse{\equal{#1}{+PRE% this comment prevents substitution
FIX+}}{\Include[TP]{#2}{#3}{#4}}{\lecturedate{\formatdate{#2}{#3}{20#4}}\input{lectures/#1-#4-#3-#2.tex}}}


%% A macro to typeset a commutitive diagram in the style of
%% \[\Abb[functionname]{from}{to}{fromelement}{toelement}\]
\newcommand{\Sidein}{\begin{rotate}{90}\small$\in$\end{rotate}}
\newcommand{\sidew}[1]{\rotatebox{90}{\small$#1$}}

\newcommand{\Abb}[5][]{\ensuremath{
    \begin{array}{lc}
      \ifthenelse{\equal{#1}{}}{}{#1:}\;\; &
      \begin{xy}
        \xymatrixrowsep{1em}\xymatrixcolsep{2em}%
        \xymatrix{ #2 \ar[r] \ar@{}[d]^<<<<{\hspace{0.001em} \Sidein}
          & #3  \ar@{}[d]^<<<<{\hspace{0.001em} \Sidein} \\
          #4 \ar@{|->}[r] & #5} \end{xy}
    \end{array}
  }%
}


%% Use the alternative epsilon per default and define the old one as \oldepsilon
\let\oldepsilon\epsilon

\renewcommand{\epsilon}{\ensuremath\varepsilon}

%% Also set the alternate phi as default.
\renewcommand{\phi}{\ensuremath{\varphi}}

%% Table centering header
\newcommand{\ctabletitle}[2]{\multicolumn{#1}{c}{\bfseries#2}}

%% emphasized equation
\newcommand{\emphequation}[2]{
	\begin{empheq}[box=\shadowbox*]{#1}
		#2
	\end{empheq}%
}

% renew \abs so the bars are adjusted to the height of the content
\renewcommand{\abs}[1]{\ensuremath{\left|#1\right|}}

% conditional equation with large left curly brace
\newcommand{\conditional}[1]{
\left\{\begin{array}{@{}l@{\quad}l}
	#1
\end{array}\right.
}

% parentheses
\newcommand{\parens}[1]{\ensuremath{\left(#1\right)}}

% 1/2 fractions
\newcommand{\half}{\ensuremath{\frac{1}{2}}}
\newcommand{\nicehalf}{\ensuremath{\nicefrac{1}{2}}}
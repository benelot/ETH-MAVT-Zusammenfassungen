%!TEX root = ../Thermodynamik I.tex

\section{Definitionen \& Grundlagen} % (fold)
	
	\subsection{Einheiten} % (fold)
		\begin{align*}
			\SI{1}{\BAR} &= \SI{e5}{\pascal} = \SI{760}{\mmHg} \\
			\SI{1}{\mega\pascal} &= \SI{10}{\BAR} = \SI[fraction=nicefrac]{1}{\newton\per\Square\millimetre} \\
			\SI{1}{\pascal} &= \SI[fraction=nicefrac]{1}{\newton\per\Square\metre} \\
			\SI{1}{atm} &= \SI{1.01325}{\BAR}
		\end{align*}
	% subsection: Einheiten (end)
	
	\subsection{Definitionen} % (fold)
		\begin{description}
			\item[Dichte] $\rho = \frac{m}{V} \nicesunit{\kg\per\cubic\metre}$
			\item[spez. Volumen] $v = \frac{1}{\rho} = \frac{V}{m} \nicesunit{\cubic\metre\per\kg}$
			\item[mol. Volumen (Molvol)] $\bar{v} = \frac{M}{\rho} = v\cdot M \nicesunit{\cubic\metre\per\kmol}$
		\end{description}
	% subsection: Definitionen (end)
	
	\subsection{Thermodynamisches System} % (fold)
		\subsubsection{Begriffe} % (fold)
			\begin{tabular*}{\textwidth}{@{}r@{\textbf{:}\hspace{2mm}}l@{}}
				\textbf{Umgebung} & alles ausserhalb des Systems \\
				\textbf{Systemgrenze} & gedachte oder materielle Begrenzung
			\end{tabular*}
		% subsubsection: Begriffe (end)
		\subsubsection{Arten} % (fold)
			\paragraph{Bezüglich Massenstrom:} % (fold)
			(Symbol: \ding{221})
			
				\begin{tabular*}{\textwidth}{@{}r@{\hspace{2mm}}l@{}}
					\textbf{geschlossen:} & kein Massenstrom durch Systemgrenze \\ & $\Rightarrow \text{ Masse} = \text{const.}$ \\
					\textbf{offen:} & Massenstrom durch Systemgrenze
				\end{tabular*}
			% paragraph: Bezüglich Massenstrom: (end)
			
			\paragraph{Bezüglich thermischer Energie:} % (fold)
				(Symbol: $\Rightarrow$)
				
				\begin{tabular*}{\textwidth}{@{}r@{\hspace{2mm}}l@{}}
					\textbf{isoliert:} & keine therm.~E.~durch Systemgrenze \\ & $\Rightarrow$ \emph{adiabat} \\
					\textbf{nicht isoliert:} & therm.~E.~durch Systemgrenze \\ & $\Rightarrow$ \emph{diatherm} od.~\emph{nicht adiabat}
				\end{tabular*}
			% paragraph: Bezüglich thermischer Energie: (end)
			
			\paragraph{Bezüglich physikalischer und chemischer Zusammensetzung:} % (fold)
				\begin{tabular*}{\textwidth}{@{}r@{\hspace{2mm}}l@{}}
					\textbf{homogenes Sys.:} & phys./chem. Zusammensetzung \\ & überall gleich \\
					\textbf{heterogenes Sys.:} & analog
				\end{tabular*}
			% paragraph: Bezüglich physikalischer und chemischer Zusammensetzung: (end)
			
		% subsubsection: Arten (end)
	% subsection: Thermodynamisches System (end)
	\subsection{Thermodynamischer Zustand} % (fold)
		\textbf{Zustandsgrössen} sind thermodyn.~Eigenschaften, die den thermodyn.~Zustand definieren. Dieser hängt \emph{nicht} vom Prozess ab.
		
		\begin{tabular*}{\textwidth}{@{}r@{\hspace{2mm}}l@{}}
			\textbf{intensive:} & unabhängig von Stoffmenge \\
			\textbf{extensive:} & abhängig von Stoffmenge \\
			\textbf{spezifische:} & extensive Zustandsgr.~pro Masseneinh. \\ & $\Rightarrow$ \emph{intensiv} \\
			\textbf{molare:} & Zustandsgr. pro Kilomol $\left(\si[per=reciprocal]{\per\kmol}\right)$ 
		\end{tabular*}
	% subsection: Thermodynamischer Zustand (end)
% section: Definitionen \& Grundlagen (end)
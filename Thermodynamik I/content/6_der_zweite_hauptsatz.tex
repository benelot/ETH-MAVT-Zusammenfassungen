 %!TEX root = /Users/philipe/Documents/ETH/3. Semester/Thermodynamik/Neue Zusammenfassung/Thermodynamik I - Zusammenfassung.tex

\section{Der zweite Hauptsatz} % (fold)
	\subsection{Formulierungen des zweiten Hauptsatzes} % (fold)
		\paragraph{Clausius:} % (fold)
			Wärme kann nicht von selbst (spontan) von einem Körper mit tieferer Temperatur auf einen Körper mit höherer Temperatur übertragen werden.
		% paragraph: Clausius: (end)
		
		\paragraph{Kelvin-Planck:} % (fold)
			Es ist unmöglich eine Maschine zu bauen, welche in einem thermischen
			Kreisprozess kontinuierlich Arbeit an die Umgebung abgibt und dabei nur
			in Kontakt mit einem einzigen Wärmereservoir steht, aus welchem es diese
			Wärme bezieht.
		% paragraph: Kelvin-Planck: (end)
	% subsection: Formulierungen des zweiten Hauptsatzes (end)
	
	\subsection{Reversible und Irreversible Prozesse} % (fold)
		Ein Prozess ist reversibel, wenn ein Ausgangszustand im System und allen Teilen der Umgebung wieder hergestellt werden kann, \emph{ohne dass eine Ver\-än\-de\-rung zu\-rück\-bleibt}. Arbeit wird ab\-ge\-führt und ohne Verlust gespeichert.
		
		In der Natur treten grundsätzlich nur \emph{irreversible Prozesse} auf. Irreversible Prozesse können im Grenzfall als einen reversiblen Prozess verstanden werden.
		
		Irreversibilitäten verursachen unerwünschte Verluste und verkleinern den Wirkungsgrad von Prozessen. Irreversibilitäten müssen erkannt und minimiert werden.
	% subsection: Reversible und Irreversible Prozesse (end)
	
	\subsection{Energiebilanz des irreversiblen Prozesses} % (fold)
		Reversible und irreversible Prozesse sind im $p$-$V$-Diagramm identisch. Es gilt in beiden Fällen:
		\begin{gather*}
			\Delta U_{\text{rev}} = Q_{\text{rev}} - W_{\text{rev}} \qquad
			\Delta U_{\text{irr}} = Q_{\text{irr}} - W_{\text{irr}} \\
			\Delta U_{\text{rev}} = \Delta U_{\text{irr}}
		\end{gather*}
		\begin{multicols}{2}
			\subsubsection{Expansion} % (fold)
				\begin{gather*}
					\abs{Q_{\text{irr}}} < \abs{Q_{\text{rev}}} \\
					\abs{W_{\text{irr}}} < \abs{W_{\text{rev}}}
				\end{gather*}
				\paragraph{Reversibel:} % (fold)
					\[
						W_{\text{rev}} = \int_{V_1}^{V_2} p \cdot \diff V
					\]
				% paragraph: Reversible Expansion: (end)
				\paragraph{Irreversibel:} % (fold)
					\[
						W_{\text{irr}} = \underbrace{\eta_{\text{i}}}_{\mathclap{\text{innerer Wirkungsgrad }< 1}} \cdot W_{\text{rev}}
					\]
				% paragraph: Irreversible Expansion: (end)
			% subsubsection: Expansion (end)
			\subsubsection{Kompression} % (fold)
				\begin{gather*}
					\abs{Q_{\text{irr}}} > \abs{Q_{\text{rev}}} \\
					\abs{W_{\text{irr}}} > \abs{W_{\text{rev}}}
				\end{gather*}
				\paragraph{Reversibel:} % (fold)
					\[
						W_{\text{rev}} = \int_{V_1}^{V_2} p \cdot \diff V < 0
					\]
				% paragraph: Reversible Kompression: (end)
				\paragraph{Irreversibel:} % (fold)
					\[
						W_{\text{irr}} = \frac{W_{\text{rev}}}{\eta_{\text{i}}} < 0 \vphantom{\underbrace{\eta_{\text{i}}}_{\mathclap{\text{innerer Wirkungsgrad }< 1}}}
					\]
				% paragraph: Irreversible Kompression: (end)
			
			% subsubsection: Kompression (end)
		\end{multicols}
	% subsection: Energiebilanz des irreversiblen Prozesses (end)
	
	\subsection{Der Kreisprozess nach Carnot} % (fold)
		Der Carnot-Prozess ist ein idealisierter reversibler Kreisprozess, der zwischen zwei Temperaturen $T_\text{C},\,T_\text{H}$ arbeitet, und dient als \emph{Definition des theoretisch maximalen Wirkungsgrades eines reversiblen Prozesses}.
		
		% TODO: Carnot-Prozess Grafik
		
		Die geleistete Arbeit ist die eingeschlossene Flä\-che. Der Durchlauf im Gegenuhrzeigersinn entspricht einer Wär\-me\-pum\-pe.
		
		Der Prozess besteht aus \emph{4 reversiblen Teilprozessen}:
		\begin{description}
			\item[$\mathbf{1 \to 2}$:] System thermisch isoliert, \textbf{adiabate Kompression} \\ $\Rightarrow$ Temperatur steigt von $T_\text{C}$ auf $T_\text{H}$
			\item[$\mathbf{2 \to 3}$:] System mit Wärmereservoir $T_\text{H}$ verbunden, \textbf{isotherme Expansion} \\ $\Rightarrow$ Wärme aus Reservoir bezogen
			\item[$\mathbf{3 \to 4}$:] System thermisch isoliert, \textbf{adiabate Expansion} \\ $\Rightarrow$ Temperatur sinkt von $T_\text{H}$ auf $T_\text{C}$
			\item[$\mathbf{4 \to 1}$:] System mit Wärmereservoir $T_\text{C}$ verbunden, \textbf{isotherme Kompression} \\ $\Rightarrow$ Wärme an Reservoir abgegeben
		\end{description}
		
		\paragraph{Beim idealen Gas:} % (fold)
			% \begin{align*}
			% 	1 \xrightarrow{Q=0} 2 \xrightarrow{Q=W} 3 \xrightarrow{Q=0} 4 \xrightarrow{Q=W} 1
			% \end{align*}
			\begin{align*}
				Q_{12} &= 0 \qquad W_{12} = m(u_{T_\text{C}} - u_{T_\text{H}}) = -W_{34} \boldsymbol{ < 0} \\
				Q_{23} &= W_{23} = (\eta - 1)^{-1} \cdot Q_{41} \\
				Q_{34} &= 0 \qquad W_{34} = m(u_{T_\text{H}} - u_{T_\text{C}}) = -W_{12} \\
				Q_{41} &= W_{41} = (\eta - 1) \cdot Q_{23}  \boldsymbol{ < 0}
			\end{align*}
			\[
				u_1 = u_4 = u_{T_\text{C}} \qquad 
				u_2 = u_3 = u_{T_\text{H}}
			\]
			
		% paragraph: Beim idealen Gas: (end)
		\paragraph{Schlussfolgerungen:} % (fold)
			\[
				\eta_\text{th} = \frac{W_\text{KP,th}}{Q_\text{H}} = \frac{Q_\text{H} - Q_\text{C}}{Q_\text{H}} = 1 - \frac{Q_\text{C}}{Q_\text{H}} < 1
			\]
			\begin{itemize}
				\item Es ist \emph{unmöglich} durch einen Kreisprozess Wärme vollständig in Arbeit umzuwandeln.
				\item Alle reversiblen Wärmekraft-Prozesse, die zwischen zwei identischen thermischen Reservoirs arbeiten, haben gleiches $\eta_\text{th}$.
				\item $\eta_\text{th}$ eines irreversiblen Wärmekraft-Prozesses ist immer kleiner als das eines reversiblen zwischen den gleichen thermischen Reservoirs.
			\end{itemize}
		% paragraph: Schlussfolgerungen: (end)
	% subsection: Der Kreisprozess nach Carnot (end)
	\subsection{Maximale Wirkungsgrade von Kreisprozessen} % (fold)
		\[
			\frac{Q_\text{C}}{Q_\text{H}} = \frac{T_\text{C}}{T_\text{H}}
		\]
		\subsubsection{Wärmekraftprozesse (Wirkungsgrad)} % (fold)
			Der \emph{Carnot-Wirkungsgrad} stellt das \emph{theoretische Maximum} dar, welches ein Kreisprozess erreichen kann.
			\begin{gather*}
				\eta_{\text{Carnot}} = 1 - \frac{Q_\text{C}}{Q_\text{H}} = 1 - \frac{T_\text{C}}{T_\text{H}} = \frac{T_\text{H} - T_\text{C}}{T_\text{H}}
			\end{gather*}
		% subsubsection: Wärmekraftprozesse (end)
		\subsubsection{Kältemaschinen und Wärmepumpen (Leistungsziffer)} % (fold)
			Bei der Kältemaschine ist man an der Wärmemenge $Q_\text{C}$ interessiert, die dem kalten Reservoir entzogen wird:
			\[
				\epsilon_{\text{KM}} = \frac{Q_\text{C}}{W_{\text{KP}}} = \frac{Q_\text{C}}{Q_\text{H} - Q_\text{C}} = \frac{T_\text{C}}{T_\text{H} - T_\text{C}}
			\]
			Bei der Wärmepumpe ist man an der Wärmemenge $Q_\text{H}$ interessiert, die auf der heissen Seite abgegeben wird:
			\[
				\epsilon_{\text{WP}} = \frac{Q_\text{H}}{W_{\text{KP}}} = \frac{Q_\text{H}}{Q_\text{H} - Q_\text{C}} = \frac{T_\text{H}}{T_\text{H} - T_\text{C}} = 1 + \epsilon_{\text{KM}}
			\]
		% subsubsection: Kältemaschinen und Wärmepumpen (end)
	% subsection: Maximale Wirkungsgrade von Kreisprozessen (end)
	
	\subsection{Die Clausius-Ungleichung und der Entropiebegriff} % (fold)
		Die \textbf{Entropie} ist
		\begin{itemize}
			\item ein Mass für die Irreversibilität
			\item \emph{keine} Erhaltungsgrösse
			\item eine Zustandsfunktion (thermodynamisches Potential)
			\item in der mikroskopischen Betrachtung ein Mass für die Unordnung
		\end{itemize}
		Die Entropieänderung ist \emph{unabhängig} vom Weg.
		
		\begin{gather*}
			\text{Irreversibilität} = \text{Entropiezuwachs } \Delta S \\
			= \frac{\text{dem Reservoir zugeführte Wärmemenge}}{\text{Temperatur des Reservoirs}} \\
			\Delta S = \frac{\Delta Q}{T} = S_2 - S_1 \nicesunit{\kilo\joule\per\kelvin} \qquad
			s = \nicefrac{S}{m} \nicesunit{\kilo\joule\per\kg\per\kelvin}
		\end{gather*}
		
		\paragraph{Clausius-Ungleichung:} % (fold)
			\[
				\oint_{\mathclap{\text{\raisebox{-2mm}{Kreisprozess}}}} \frac{\delta Q}{T} \leq 0 \qquad \Rightarrow \quad
				\frac{Q_\text{H}}{T_\text{H}} \leq \frac{Q_\text{C}}{T_\text{C}}
			\]
			wobei das Gleichheitszeichen nur für reversible Prozesse gilt.
			Der Betrieb einer Maschine ist mit einer Entropiezunahme verbunden.
			\emph{Die Maschine kann keine Entropie vernichten.}
		% paragraph: Clausius-Ungleichung: (end)
		
		\paragraph{Zusammenfassend:} % (fold)
			Ein Prozess wird spontan immer in der Richtung ablaufen, sodass die Entropie zunimmt.
			
			Die Entropie ist ein Mass dafür, wieviel von der potentiellen Arbeitsmöglichkeit einer thermischen Energiemenge schon verloren/verbraucht) ist.
		% paragraph: Zusammenfassend: (end)
	% subsection: Die Clausius-Ungleichung und der Entropiebegriff (end)
	
	\subsection{Entropie einer reinen einfachen Substanz} % (fold)
		Absolutwert der Entropie für einen Zustand $y$ mit Referenzzustand $x$:
		\[
			S_y = S_x + \int_x^y \frac{\delta Q_{\text{rev}}}{T}
		\]
	
		\subsubsection{Tabellenwerte:} % (fold)
			\begin{center}
				\begin{tabular}{ll}
					\toprule
					Gebiet & Entropie $s$ \\
					\midrule
					überhitzter Dampf & $s(T,p)$ \\
					Sättigungslinie & $s(T_{\text{sat}})$ oder $s(p_{\text{sat}})$ \\
					Nassdampfgebiet & $s(x,T) = s_f + x \cdot (s_g - s_f)$ \\
					unterkühlte Flüssigkeit & $s(T,p) \approx s_f(T)$ \\
					\bottomrule
				\end{tabular}
			\end{center}
		% subsubsection: Tabellenwerte: (end)
	
		\subsubsection{TdS-Gleichungen} % (fold)
			Gelten auch für irreversible Prozesse.
			%\setlength{\mathindent}{.5\mathindent}
			\begin{multicols}{2}
				\paragraph{1. TdS-Gleichung:} % (fold)
					\setlength{\mathindent}{.5\mathindent}
					\begin{empheq}[box=\shadowbox]{align*}
						T\cdot \diff S &= \diff U + p \cdot \diff V \\
						T\cdot \diff s &= \diff u + p \cdot \diff v \\
						T\cdot \diff \overline{s} &= \diff \overline{u} + p \cdot \diff \overline{v}
					\end{empheq}
				% paragraph: 1. TdS-Gleichung: (end)
				\paragraph{2. TdS-Gleichung:} % (fold)
					\begin{empheq}[box=\shadowbox]{align*}
						T\cdot \diff S &= \diff H - V \cdot \diff p \\
						T\cdot \diff s &= \diff h - v \cdot \diff p \\
						T\cdot \diff \overline{s} &= \diff \overline{h} - \overline{v} \cdot \diff p
					\end{empheq}
				% paragraph: 2. TdS-Gleichung: (end)
			\end{multicols}
			%\setlength{\mathindent}{2\mathindent}
		% subsubsection: TdS-Gleichungen (end)
	
		\subsubsection{Entropieänderung idealer Gase} % (fold)
		
			\begin{align*}
				s(T_2,v_2) - s(T_1,v_1) &= c_v \cdot \ln \frac{T_2}{T_1} + R \cdot \ln \frac{v_2}{v_1} \\
				s(T_2,p_2) - s(T_1,p_1) &= c_p \cdot \ln \frac{T_2}{T_1} - R \cdot \ln \frac{p_2}{p_1} \\
				&= \underbrace{s_2\!\!\degree - s_1\!\!\degree}_{\mathclap{\text{aus Tabellen}}} - R \cdot \ln \frac{p2}{p1}
			\end{align*}
	
			% \begin{gather*}
			% 	\diff s = c_v(T) \cdot \frac{\diff T}{T} + R \cdot \frac{\diff v}{v} \qquad
			% 	\diff s = c_p(T) \cdot \frac{\diff T}{T} - R \cdot \frac{\diff p}{p}
			% \end{gather*}
			% Zwischen zwei Zuständen integriert:
			% \begin{align*}
			% 	
			% \end{align*}
		% subsubsection: Entropieänderung idealer Gase (end)
	
		\subsubsection{Entropieänderung inkompressibler Stoffe} % (fold)
			\begin{align*}
				v = \const \quad \Rightarrow \  & c_p = c_v = c(T) \\
				& \diff u = \diff h = c(T) \cdot \diff T \\
				& \diff s = c(T) \cdot \frac{\diff T}{T}
			\end{align*}
			\begin{empheq}[box=\shadowbox]{align*}
				s_2 - s_1 &= \int_{T_1}^{T_2} c(T) \cdot \frac{\diff T}{T} \\
				&\stackrel{\mathclap{c = \const}}{=} \quad c \cdot \ln \frac{T_2}{T_1}
			\end{empheq}
		% subsubsection: Entropieänderung inkompressibler Stoffe (end)
	% subsection: Entropie einer reinen einfachen Sunstan (end)
	
	\subsection{Entropiebilanz für geschlossene Systeme} % (fold)
		
		\subsubsection{Entropieänderung bei reversiblen Prozessen} % (fold)
			\emph{Ein reversibler adiabatischer Prozess ist gleichzeitig ein isentroper Prozess.}
			
			Bei Wärmezufuhr nimmt die Entropie des Systems zu, bei Wär\-me\-ab\-fuhr ab.
			\[
				\diff S = \left(
					\frac{\delta Q}{T}
				\right)_{\text{rev}} \qquad S_{\text{erz}} = 0
			\]
		% subsubsection: Entropieänderung bei reversiblen Prozessen (end)
		
		\subsubsection{Entropie-Produktion} % (fold)
			\begin{empheq}[box=\shadowbox*]{equation*}
				S_{\text{erz}} = S_2 - S_1 - \sum_{j=1}^n \frac{Q_j}{T_j}
			\end{empheq}
			
			\paragraph{Achtung:} % (fold)
				$S_{\text{erz}}$ ist \emph{keine Zustandsgrösse}.
			% paragraph: Achtung: (end)
			
			\paragraph{Bei Kreisprozessen} % (fold)
				ist $S_1 = S_2$ und somit
				\begin{empheq}[box=\shadowbox*]{equation*}
					S_{\text{erz,KP}} = - \sum_{j=1}^n \frac{Q_j}{T_j}
				\end{empheq}
			% paragraph: Bei Kreisprozesse (end)
			
			% \paragraph{Differentielle Betrachtung:} % (fold)
			% 	\[
			% 		S_{\text{erz}} = S_2 - S_1 - \int_1^2 \left(
			% 			\frac{\delta Q}{T}
			% 		\right)_{\text{G}}
			% 	\]
			% % paragraph: Differentielle Betrachtung: (end)
		% subsubsection: Entropie-Produktion (end)
		
	% subsection: Entropiebilanz für geschlossene Systeme (end)
	
	\subsection{Entropiebilanz für offene Systeme} % (fold)
		\begin{empheq}[box=\shadowbox]{equation*}
			\underbrace{\vphantom{\sum_j^l}
				\dot S_{\text{erz}}
			}_{\text{\ding{172}}}
			=
			\underbrace{\vphantom{\sum_j^l}
				\Diff{S}{t}
			}_{\text{\ding{173}}}
			-
			\underbrace{\vphantom{\sum_j^l}
				\sum_{i=1}^l \frac{\dot Q_i}{T_i}
			}_{\text{\ding{174}}}
			+
			\underbrace{\vphantom{\sum_j^l}
				\sum_{j=1}^m \dot m_{j,a} \cdot s_{j,a}
			}_{\text{\ding{175}}}
			-
			\underbrace{\vphantom{\sum_j^l}
				\sum_{k=1}^n \dot m_{k,e} \cdot s_{k,e}
			}_{\text{\ding{176}}}
		\end{empheq}
		\begin{enumerate}
			\item Erzeugungsrate im System
			\item Zunahme des Entropieinhaltes des Systems
			\item Entropietransport über Systemgrenze durch Wär\-me\-lei\-tung
			\item mit Masse ausströmender Entropiestrom
			\item mit Masse einströmender Entropiestrom
		\end{enumerate}
		
		\paragraph{Bei stationären Prozessen:} % (fold)
			\begin{empheq}[box=\shadowbox]{equation*}
				\dot S_{\text{erz}}
				=
				-\sum_{i=1}^l \frac{\dot Q_i}{T_i}
				+
				\dot m \cdot (
					s_a - s_e
				)
			\end{empheq}
		% paragraph: Bei stationären Prozessen (end)
	% subsection: Entropiebilanz für offene Systeme (end)
	
	\subsection{Isentrope Prozesse} % (fold)
		Prozess bei \emph{konstanter Entropie} ($\Delta S = 0$). Es darf keine Wärme übertragen werden und der Prozess muss reversibel sein $\Rightarrow$ \emph{adiabatischer reversibler Prozess}
		
		\subsubsection{Der isentrope Prozess beim idealen Gas} % (fold)
			\begin{empheq}[box=\shadowbox]{equation*}
				\frac{p_2}{p_1} = \left(
					\frac{v_1}{v_2}
				\right)^{\kappa}
			\end{empheq}
			Vgl.~Abschnitt~\emph{\ref{subs:polytrope_zustandsaenderung}~\nameref{subs:polytrope_zustandsaenderung}} auf Seite~\pageref{subs:polytrope_zustandsaenderung} mit $\boldsymbol{n = \kappa = }\nicefrac{\boldsymbol{c_p}}{\boldsymbol{c_v}}$
		% subsubsection: Der isentrope Prozess beim idealen Gas: (end)
		
		\subsubsection{Der isentrope Wirkungsgrad} % (fold)
			\[
				\eta_{\text{isentrop}} = \frac{\text{Arbeitleistung des realen Prozesses}}{\text{Arbeitsleistung des ideal reversiblen Prozesses}}
			\]
			wobei die Arbeitsleistung des ideal reversiblen Prozesses bezüglich identischer Ein-/Austrittsbedingungen ist.
			
			\paragraph{Turbine:} % (fold)
				\[
					\eta_{\text{T},s} = \frac{h_1 - h_2}{h_1 - h_{2,s}} = \frac{\nicefrac{\dot W}{\dot m}}{\nicefrac{\dot W_{\text{max}}}{\dot m}}
				\]
			% paragraph: Turbine: (end)
			\paragraph{Düse:} % (fold)
				\[
					\eta_{\text{D},s} = \frac{h_2 - h_1}{h_{2,s} - h_1} = \frac{\nicefrac{w_2^2}{2}}{\nicefrac{w_{2\text{,max}}^2}{2}}
				\]
			% paragraph: Düse: (end)
			\paragraph{Kompressor/Pumpe:} % (fold)
				\[
					\eta_{\text{K/P},s} = \frac{h_{2,s} - h_1}{h_2 - h_1} = \frac{\nicefrac{\dot W_{\text{min}}}{\dot m}}{\nicefrac{\dot W}{\dot m}}
				\]
			% paragraph: Kompressor/Pumpe: (end)
			
			\paragraph{Berechnung der isentropen Enthalpie:}~\\ % (fold)
				Es gilt $s_1 = s_\text{$2$,s}$.
				
				\textbf{$\boldsymbol{(v_1 \approx v_2)}$:}
				\[
					h_\text{$2$,s} - h_1 = \int \diff h = \int \parens{ T \diff s + v \diff p} = \cancelto{0}{T\Delta s} + v \Delta p
				\]
				
				\textbf{Sonst:}
				$h_\text{$2$,s}$ dort entnehmen, wo die Entropie gleich $s_1 = s_\text{$2$,s}$ ist.
			% paragraph: Berechnung der isentropen Enthalpie: (end)
		% subsubsection: Der isentrope Wirkungsgrad (end)
		
		\subsubsection{Arbeit und Wär\-me\-über\-tra\-gung bei sta\-tio\-nä\-ren, intern-reversiblen Prozessen} % (fold)
			\paragraph{Isotherme reversible Prozesse:} % (fold)
				$\mathbold{\dot S_{\text{erz}} = 0}$
				Über\-tra\-ge\-ne Wär\-me\-men\-ge (pro Masseneinheit):
				\[
					\frac{\dot Q}{\dot m} = T(s_2 - s_1)
				\]
				Somit lautet der 1.~Hauptsatz für dieses offene System mit $\mathbold{\Delta u = 0}$:
				%\begin{nomathindent}
					\begin{empheq}[box=\shadowbox]{equation*}
						\frac{\dot W_{\text{rev}}}{\dot m}
						=
						\underbrace{\vphantom{\bigg(}
							T(s_2-s_1)
						}_{\nicefrac{\dot Q}{\dot m}}
						\!-\!
						\underbrace{\vphantom{\bigg(}
							\!\left(
								p_2 \frac{\dot V_2}{\dot m} \!-\! p_1 \frac{\dot V_1}{\dot m}
							\right)
						}_{=\:0 \text{ für IG}}
						\!-\!
						\underbrace{\vphantom{\bigg(}
							\frac{w_2^2-w_1^2}{2}
						}_{\nicefrac{\Delta KE}{\dot m}}
						\!-\!
						\underbrace{\vphantom{\bigg(}
							g(z_2 - z_1)
						}_{\nicefrac{\Delta PE}{\dot m}}
					\end{empheq}
				%\end{nomathindent}
			% paragraph: Isotherm: (end)
			\paragraph{Allgemein reversible Prozesse:} % (fold)
				Wärmeübertragung:
				\[
					\frac{\dot Q_{\text{rev}}}{\dot m} = \int_1^2 T \cdot \diff s
				\]
				1. Hauptsatz mit $\mathbold{\Delta u \neq 0}$:
				\begin{empheq}[box=\shadowbox]{equation*}
					\frac{\dot W_{\text{rev}}}{\dot m}
					=
					-
					\int_1^2 v \cdot \diff p
					-
					\frac{w_2^2-w_1^2}{2}
					-
					g(z_2 - z_1)
				\end{empheq}
			% paragraph: Allgemein reversible Prozesse: (end)
			
			\paragraph{Anwendungen} % (fold)
				\subparagraph{Düsen und Diffusoren:} % (fold)
					$\dot W = 0,\, \Delta PE = 0$
					\[
						\frac{w_2^2-w_1^2}{2} = - \int_1^2 v \cdot \diff p
					\]
				% subparagraph: Düsen und Diffusoren: (end)
				\subparagraph{Turbinen, Kompressoren und Pumpen:} % (fold)
					$\Delta KE = \Delta PE = 0$
					\[
						\frac{\dot W_{\text{rev}}}{\dot m} = - \int_1^2 v \cdot \diff p
					\]
				% subparagraph: Turbinen, Kompressoren und Pumpen: (end)
			% paragraph: Anwendungen: (end)
			\paragraph{Ausführung des Integrals} % (fold)
				$\int_1^2 v \cdot \diff p$
				\[
					-\int_1^2 v \cdot \diff p = -(\const)^{\nicefrac{1}{n}} \int_1^2 \frac{\diff p}{p^{\nicefrac{1}{n}}}
				\]
				\subparagraph{für $\mathbold{n\neq 1}$:} % (fold)
					\[
						-\int_1^2 v \cdot \diff p = \frac{n}{n-1} (p_2 \cdot v_2 - p_1 \cdot v_1)
					\]
				% subparagraph: $\mathbold{n\neq 1}$ (end)
				\subparagraph{für $\mathbold{n = 1}$ (isotherm):} % (fold)
					\[
						-\int_1^2 v \cdot \diff p = -p_1 \cdot v_1 \cdot \ln \frac{p_2}{p_1}
					\]
				% subparagraph: für $\mathbold{n = 1}$: (end)
				\subparagraph{für ideales Gas mit $\mathbold{n\neq 1}$:} % (fold)
					\begin{align*}
						-\int_1^2 v \cdot \diff p &= -\frac{n}{n-1} \cdot R \cdot (T_2 - T_1) \\
						&= -\frac{n}{n-1} \cdot R \cdot T_1 \cdot \left[
							\left(
								\frac{p_2}{p_1}
							\right)^{\frac{n-1}{n}}
							-1
						\right]
					\end{align*}
				% subparagraph: $\mathbold{n\neq 1}$ (end)
				\subparagraph{für ideales Gas mit $\mathbold{n = 1}$ (isotherm):} % (fold)
					\[
						-\int_1^2 v \cdot \diff p = -R \cdot T_1 \cdot \ln \frac{p_2}{p_1}
					\]
				% subparagraph: für $\mathbold{n = 1}$: (end)
			% paragraph: Ausführung des Integrals $\int_1^2 v \cdot \diff p$ (end)
		% subsubsection: Arbeit und Wärmeübertragung bei stationären, intern-reversiblen Prozessen (end)
	% subsection: Isentrope Prozesse (end)
% section: Der zweite Hauptsatz (end)
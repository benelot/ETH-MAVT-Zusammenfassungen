%!TEX root = ../Thermodynamik I.tex

\section{Nutzbarkeit der Energie -- Exergie} % (fold)
	
	\subsection{Definition} % (fold)
		Energie setzt sich aus \textbf{Exergie} und \textbf{Anergie} zusammen:
		\begin{description}
			\item[Exergie:] Maximaler Anteil, der mittels eines Prozesses in Arbeit
				umgewandelt werden kann, wenn das System bis zum vollständigen
				Gleichgewicht mit der Umgebung betrieben wird.
			\item[Anergie:] Anteil, der nach dem Erreichen des Gleichgewichtes zurückbleibt.
		\end{description}
		
		Ein System im Gleichgewicht besitzt nur noch Anergie. Es befindet sich in
		einem ``thermodynamischen toten Zustand'' $\Rightarrow$ Entropie ist maximal
		
		\subsubsection{Exergie eines geschlossenen Systems} % (fold)
			(Index $0$ steht für Umgebung)
			\begin{align*}
				E_x &= W_\text{nutz,rev} \\
				&= U - U_0 + p_0 (V - V_0) - T_0 (S-S_0) + KE + PE \\
				e_x &= u - u_0 + p_0 (v - v_0) - T_0(s-s_0) + ke + pe
			\end{align*}
			Die Exergie ist eine \emph{Zustandsfunktion}. Für die Änderung gilt:
			\begin{align*}
				E_{x2} - E_{x1} &= (U_2 - U_1) + p_0(V_2 - V_1) - T_0(S_2 - S_1)\\ &\phantom{=} + (KE_2 - KE_1) + (PE_2 - PE_1) \\[1ex]
				e_{x2} - e_{x1} &= (u_2 - u_1) + p_0(v_2 - v_1) - T_0(s_2 - s_1)\\ &\phantom{=} + (ke_2 - ke_1) + (pe_2 - pe_1)
			\end{align*}
		% subsubsection: Exergie eines geschlossenen Systems (end)
		
		\subsubsection{Das Gouy-Stodola-Theorem für geschlossene Systeme} % (fold)
			Stellt eine Verbindung zwischen der Entropieerzeugung und verlorener Exergie dar.
			\begin{align*}
				\dot W_\text{Verlust} = \dot W_\text{rev} - \dot W_0 &= T_0 \cdot \dot S_\text{Erz} \\
				\dot E_{x\text{,Verlust}} = \dot E_{x\text{,rev}} - \dot E_{x} &= T_0 \cdot \dot S_\text{Erz} \\
			\end{align*}
		% subsubsection: Das Gouy-Stodola-Theorem für geschlossene Systeme (end)
	% subsection: Definition (end)
	
	\subsection{Exergiebilanz für geschlossene Systeme} % (fold)
		Exergie kann durch Wärme oder Arbeit übertragen werden.
		
		Exergieänderung des Systems:
		{\setlength{\mathindent}{0pt}
		\begin{align*}
			\Delta E_{x} = 
			\underbrace{
				\int_1^2 \parens{ 1- \frac{T_0}{T} } \delta Q
			}_{\mathclap{\substack{
				\text{Exergietransfer} \\
				\text{durch Wärmeübertragung}
			}}}
			 - 
			\overbrace{\vphantom{\int}
				[W - p_0(V_2 - V_1)]
			}^{\mathclap{\substack{
				\text{Exergietransfer durch} \\
				\text{Übertragung von Arbeit}
			}}}
			 - 
			\underbrace{\vphantom{\int}
				T_0 \cdot S_\text{Erz}
			}_{\mathclap{\substack{
				\text{Exergieverlust durch} \\
				\text{Entropieerzeugung}
			}}}
		\end{align*}}
		
		Bilanz für Energieströme:
		\emphequation{equation*}{
			\Diff{E_x}{t} = \sum_{i=1}^n \parens{
				1- \frac{T_0}{T_i}
			} \dot Q_i - \left[
				\dot W - p_0 \Diff{V}{t}
			\right] - T_0 \cdot \dot S_\text{Erz}
		}
		
		\paragraph{Exergieverlust beim Wärmedurchgang durch eine Wand}~\\ % (fold)
			Wärme $\delta Q$ am Eintritt und Austritt der Wärmetauscherwand ist die selbe.
			\[
				E_{xQ1} - E_{xQ2} = T_0 \frac{T_1 - T_2}{T_1 \cdot T_2} \delta Q
			\]
		% paragraph: Exergieverlust beim Wärmedurchgang durch eine Wand (end)
		
		\paragraph{Exergieverlust in abgeschlossenen (isolierten) Systemen}~\\ % (fold)
			$Q=W=0$ $\Rightarrow$
			\[
				(E_{x2} - E_{x1})_\text{abgeschl.} = - T_0 \cdot S_\text{Erz}
			\]
		% paragraph: Abgeschlossene Systeme (end)
		
		\subsubsection{Wirkungsgrad irreversibler Kreisprozesse} % (fold)
			\[
				\eta = \underbrace{1- \frac{T_\text{C}}{T_\text{H}}}_{\mathclap{\eta_\text{Carnot}}} - \frac{T_\text{C} \cdot E_{x\text{,Verlust}}}{T_0\cdot Q_\text{H}}
			\]
		% subsubsection: Wirkungsgrad irreversibler Kreisprozesse (end)
	% subsection: Exergiebilanz für geschlossene Systeme (end)
	
	\subsection{Exergiebilanz für offene Systeme} % (fold)
		Für \textbf{reversible} Prozesse:
		\emphequation{align*}{
			\Diff{E_x}{t} &= \sum_i \dot m_{\text{ein,}i} \cdot e_{x\text{,Str,ein,}i} - 
			\sum_i \dot m_{\text{aus,}i} \cdot e_{x\text{,Str,aus,}i} \\
			&\phantom{=} - \dot W_{0\text{,rev,Nutz}} + \sum_i \dot Q_i \parens{
				1- \frac{T_0}{T_i}
			}
		}
		
		Für \textbf{irreversible} Prozesse:
		\emphequation{align*}{
			\Diff{E_x}{t} &= \sum_i \dot m_{\text{ein,}i} \cdot e_{x\text{,Str,ein,}i} - 
			\sum_i \dot m_{\text{aus,}i} \cdot e_{x\text{,Str,aus,}i} \\
			&\phantom{=} + \sum_i \dot Q_i \parens{
				1- \frac{T_0}{T_i} - T_0 \cdot \dot S_\text{Erz}
			}
		}
		
		Für beide gilt:
		\[
			e_{x\text{,Strömung}} = (h-h_0) - T_0(s-s_0) + \half \cdot w^2 + g\cdot z
		\]
		$T_0$ ist die Umgebungstemperatur, $T_i$ die Temperatur am Ort der $Q_i$-Übertragung.
		
		Die gesamte Exergie einer strömenden Fluidmasse heisst \textbf{Strömungsexergie}:
		\[
			E_{x\text{,Strömung}} = m \cdot e_{x\text{,Strömung}}
		\]
		
		Bei \textbf{stationären Prozessen} gilt $\Diff{E_x}{t} =0$.
	% subsection: Exergiebilanz für offene Systeme (end)
	
	\subsection{Der exergetische Wirkungsgrad} % (fold)
		Der energetische Wirkungsgrad bewertet die Nutzung der Energie quantitativ während der exergetische Wirkungsgrad diese qualitativ bewertet.
		\[
			\epsilon = \frac{\text{genutzter Exergiestrom}}{\text{zugeführter Exergiestrom}} = \frac{\dot Q_N}{\dot Q_Q} \cdot \frac{1-\nicefrac{T_0}{T_N}}{1-\nicefrac{T_0}{T_Q}}
		\]
		mit den Inizes $0$ für Umgebung, $N$ für Nutz und $Q$ für Quelle.
		
		Um einen hohen exergetischen Wirkungsgrad zu erreichen muss jeder Prozess möglichst einem reversiblen Prozess angenähert werden.
		
		\subsubsection{Exergetisscher Wirkungsgrad für ein offenes System mit Arbeitsleistung} % (fold)
			(z.B.~Turbine)
			\[
				\epsilon = \frac{\dot E_{x\text{,Nutz}}}{\dot m (e_{x1\text{,Str}} - e_{x2\text{,Str}})} = \frac{\dot E_{x\text{,Nutz}}}{\dot E_{x\text{,Nutz}} + T_0 \cdot \dot S_\text{Erz}}
			\]
		% subsubsection: Exergetisscher Wirkungsgrad für ein offenes System mit Arbeitsleistung (end)
	% subsection: Der exergetische Wirkungsgrad (end)
% section: Nutzbarkeit der Energie -- Exergie (end)
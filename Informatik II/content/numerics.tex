%!TEX root = ../Informatik II.tex

\section{Numerics} % (fold)
	
	\subsection{Time Integration} % (fold)
	
		Most particle systems can be described with ODEs as:
		\[
			\Diff x t = u \quad, \qquad m \cdot \Diff u t = F
		\]
		
		Criteria to select a scheme:
		\begin{itemize}
			\item Consistency
			\item Accuracy
			\item Stability
			\item Efficiency
		\end{itemize}
		
		Important notions:
		\begin{itemize}
			\item Preservation of \textbf{time symmetries}
			\item Conservation of \textbf{physical quantities} such as $E$, $m\cdot u$
		\end{itemize}
		
		\subsubsection{The Leapfrog Scheme} % (fold)
			
			\begin{itemize}
				\item Time centered
			\end{itemize}
			
			\begin{gather*}
				\frac{x^{n+1} - x^n}{\delta t} = u^{n+\nicehalf} \\
				m \cdot \frac{u^{n+\nicehalf} - u^{n-\nicehalf}}{\delta t} = F(x^n)
			\end{gather*}
			
			In general:
			\[
				\sum_{i=0}^k \alpha_{k-i} x^{n+k-i} = \frac{\delta t^2}{m} \sum_{i=0}^k b_{k-i} F^{n+k-i}
			\]
			If $b_k$ is zero the scheme is explicit, else it is implicit and solutions are found iteratively.
		% subsubsection: The Leapfrog Scheme (end)
		
		\subsubsection{The Verlet Algorithm} % (fold)
			Use postitions $r(t)$, accelerations $a(t)$ and $r(t-\diff t)$, velocities do not appear.
			
			\begin{gather*}
				r \parens{t + \diff t} = r(t) + \diff t \cdot u \parens{t + \frac{\diff t}{2}} \\
				u \parens{t + \frac{\diff t}{2}} = u \parens{t - \frac{\diff t}{2}} + \diff t \cdot a(t)
			\end{gather*}
			
			Store $r(t)$, $a(t)$ and $u \parens{t-\frac{\diff t}{2}}$.
			\[
				u(t) = \nicehalf \left[
					u \parens{t + \frac{\diff t}{2}} + u \parens{t - \frac{\diff t}{2}}
				\right]
			\]
		% subsubsection: The Verlet Algorithm (end)
		
		\subsubsection{Euler Scheme} % (fold)
			
			\begin{itemize}
				\item \textbf{Not} time centered
			\end{itemize}
			
		% subsubsection: Euler Scheme (end)
		
	% subsection: Time Integration (end)
	
	\subsection{Algorithmic Complexity} % (fold)
		Complexity determines the spectrum of problems that can practically be solved on a given computer in a given time. The Bachmann-Landau (aka.~Big $O$) notation is used to express this scaling.
		$O$ gives the \textbf{upper bound} on the scaling.
	% subsection: Algorithmic Complexity (end)
	
% section: Numerics (end)
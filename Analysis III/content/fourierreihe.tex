%!TEX root = ../Analysis III.tex

\section{Fourierreihen} % (fold)
	
	\begin{definition}
		Die Funktion
		\emphequation{equation}{
			\label{fourier}
			f(x) = \frac{a_0}{2} + \sum_{n=1}^\infty \left[a_n \cos(nx) + b_n \sin(nx)\right]
		}
		heisst \emph{trigonometrische Reihe} ($p=2\pi$).
	\end{definition}
	
	\subsection{Orthogonalitätsrelation} % (fold)
		Für $k,n \in \N_0$ gilt:
		\begin{align*}
			\int_{-\pi}^\pi \cos(kx)\cos(nx) \diff x &= \conditional{
				2\pi & \text{falls $n=k=0$} \\
				\pi & \text{falls $n=k\neq0$} \\
				0 & \text{falls $n\neq0$}
			}
			\\[1ex]
			\int_{-\pi}^\pi \sin(kx)\sin(nx) \diff x &= \conditional{
				0 & \text{falls $n=k=0$} \\
				\pi & \text{falls $n=k\neq0$} \\
				0 & \text{falls $n\neq0$}
			}
			\\[1ex]
			\int_{-\pi}^\pi \cos(kx)\sin(nx) \diff x &= 0
			\\[1ex]
			\int_{-\pi}^\pi \eu^{\iu nx}\eu^{-\iu kx} \diff x &= \conditional{
				0 & \text{falls $n\neq k$} \\
				2\pi & \text{falls $n=k$}
			}
		\end{align*}
	% subsection: Orthogonalitätsrelation (end)
	\subsection{Fourierreihe 2\greektext{p}\latintext{}-periodischer Funktionen} % (fold)
		Die $2\pi$-periodische Funktion $f(x)$ werde durch die Reihe~\eqref{fourier}
		dargestellt. Dann gilt:
		\emphequation{align*}{
			a_n &= \frac{1}{\pi}\int_{-\pi}^\pi f(x) \cos(nx) \diff x \\[1ex]
			b_n &= \frac{1}{\pi}\int_{-\pi}^\pi f(x) \sin(nx) \diff x
		}
		
		\begin{bemerkung}
			Anstelle des Intervalls $[-\pi,\pi]$ kann jedes Intervall $[x,x+2\pi]$
			verwendet werden.
		\end{bemerkung}
		
		Können die Integrale $a_n$ und $b_n$ gebildet werden, so kann man die \emph{formale} Fourierreihe bilden.
		\[
			f(x) \approx \frac{a_0}{2} + \sum_{n=1}^\infty \left[a_n \cos(nx) + b_n \sin(nx)\right]
		\]
	% subsection: Fourierreihe 2π-periodischer Funktionen (end)
	\subsection{Fourierreihen gerader und ungerader Funktionen} % (fold)
		\begin{description}
			\item[\phantom{un}$\boldsymbol f$ gerade:] $\displaystyle\quad
				b_n = 0\ ,\quad a_n = \frac{2}{\pi}\int_0^\pi f(x) \cos(nx) \diff x
			$
			\item[$\boldsymbol f$ ungerade:] $\displaystyle\quad
				a_n = 0\ ,\quad b_n = \frac{2}{\pi}\int_0^\pi f(x) \sin(nx) \diff x
			$
		\end{description}
	% subsection: Fourierreihen gerader und ungerader Funktionen (end)
	\subsection{Komplexe Schreibweise} % (fold)
		Mit $
			\eu^{\iu nx} = \cos(nx) + \iu \sin(nx)
		$
		lässt sich $f(x)$ umformen zu
		\emphequation{gather*}{
			f(x) = \sum_{n=-\infty}^{\infty} c_n\: \eu^{\iu nx}
			\\[1ex]
			c_n =
			\frac{1}{2\pi}\int_{-\pi}^\pi f(x)\: \eu^{-\iu nx} \diff x \quad
			=\conditional{
				\frac{a_n-\iu b_n}{2} & \text{falls $n > 0$} \\[1ex]
				\frac{a_0}{2} & \text{falls $n=0$} \\[1ex]
				\frac{a_{-n} + \iu b_{-n}}{2} & \text{falls $n<0$}
			}
		}
		
		\paragraph{Umkehrformel} % (fold)
			\begin{align*}
				a_0 &= 2 c_0 \\
				a_n &= c_n + c_{-n} \qquad n>0\\
				b_n &= \iu (c_n - c_{-n})
			\end{align*}
		% paragraph: Umkehrformel (end)
	% subsection: Komplexe Schreibweise (end)
	\subsection{Fourierreihen von Funktionen mit allgemeiner Periode} % (fold)
		$f(t)$ hat Periode $T$.
		$\quad \widetilde f(x) := f(\frac{T}{2\pi}x)$ hat Periode $2\pi$.
		
		Folglich hat eine periodische Funktion mit Periode $T$ die (formale) Fourierreihe \emphequation{align*}{
			f(t) &= \frac{a_0}{2} + \sum_{n=1}^\infty \left[a_n \cos\left(n\frac{2\pi}{T}t\right) + b_n \sin\left(n\frac{2\pi}{T}t\right)\right] \\[2ex]
			a_n &= \frac{2}{T} \int_0^T f(t) \cos\left(n \frac{2\pi}{T} t\right) \diff t \\[1ex]
			b_n &= \frac{2}{T} \int_0^T f(t) \sin\left(n \frac{2\pi}{T} t\right) \diff t
		}
		\emphequation{align*}{
			f(t) &= \sum_{n=-\infty}^{\infty} c_n\: \eu^{\iu n \frac{2\pi}{T} t}
			\\[1ex]
			c_n &=
			\frac{1}{T}\int_{0}^T f(t)\: \eu^{-\iu n \frac{2\pi}{T} t} \diff x
		}
	% subsection: Fourierreihen von Funktionen mit allgemeiner Periode (end)
% section: Fourrierreihen (end)
\documentclass[9pt,fleqn,ngerman,article]{memoir}

\title{Lineare Algebra -- Zusammenfassung}
\author{Philipe Fatio}

%% Memoir layout setup

%% NOTE: You are strongly advised not to change any of them unless you
%% know what you are doing.  These settings strongly interact in the
%% final look of the document.

% Dependencies
\usepackage{array}
\usepackage{wrapfig}
\usepackage{multicol}
\usepackage[thmmarks]{ntheorem}

\columnsep 30pt
\columnseprule \normalrulethickness

% Define the default sans serif font as the lighter computer modern bright by
% D. Knuth.
\renewcommand{\sfdefault}{cmbr}

\makeatletter

% Set the way pages are layed out (headers and page numbering)
\makepagestyle{mystyle}
\makeheadrule{mystyle}{\textwidth}{\normalrulethickness}
\makeoddhead{mystyle}{\textsc{\@title}}{}{\emph{\@author} -- \today{}}
\makeevenhead{mystyle}{\textsc{\@title}}{}{\emph{\@author} -- \today{}}
\makeoddfoot{mystyle}{}{\small -- \thepage{} --}{}
\makeevenfoot{mystyle}{}{\small -- \thepage{} --}{}
\pagestyle{mystyle}

% Use the newly defined style
\chapterstyle{VZ14}

% Redefine sectional headings to contain rules
\renewcommand{\section}{\@startsection{section}{1}{0mm}%
{-2\baselineskip}{0.8\baselineskip}%
{\hrule depth 0.2pt width\columnwidth\hrule depth1.5pt
width0.25\columnwidth\vspace*{1.2em}\Large\bfseries\sffamily}}
\makeatother

\makeatletter
\renewcommand{\subsection}{\@startsection{subsection}{1}{0mm}%
{-2\baselineskip}{0.8\baselineskip}%
{\hrule depth 0.2pt width\columnwidth\hrule depth0.75pt
width0.25\columnwidth\vspace*{1.2em}\large\bfseries\sffamily}}
\makeatother

\makeatletter
\renewcommand{\subsubsection}{\@startsection{subsubsection}{1}{0mm}%
{-2\baselineskip}{0.8\baselineskip}%
{\hrule depth 0.2pt width\columnwidth\vspace*{1.2em}\normalsize\bfseries\sffamily}}

\renewcommand*{\thesection}{\arabic{section}}

\setparaheadstyle{\normalsize\bfseries\sffamily}
\setsubparaheadstyle{\normalsize\bfseries\sffamily}

% Set captions to a more separated style for clearness
\captionnamefont{\sffamily\bfseries\footnotesize}
\captiontitlefont{\sffamily\footnotesize}
\setlength{\intextsep}{16pt}
\setlength{\belowcaptionskip}{1pt}

% Set section and TOC numbering depth to subsection
\setsecnumdepth{subsection}
\settocdepth{subsection}

% Turn off american style paragraph indentation and add some space to be
% printed when a new paragraph starts.

\setlength{\parindent}{0pt}
\addtolength{\parskip}{1pt}

\setlength{\droptitle}{-2em}
\renewcommand{\maketitlehookb}{\vspace{-1em}}
\date{}

% A bit spacier tabulars and lists
\setlength{\extrarowheight}{4pt}
\setlength{\itemsep}{10pt}
% \renewcommand{\arraystretch}{1.2}

% % This provides a frontend to set the lecture date into the header
% \newcommand{\lecturedate}[1]{\def\@lecdate{#1}}
% \makeoddhead{Ruled}{\@lecdate}{}{\normalfont\rightmark}

\makeatother

% This defines how theorems should look. Best leave as is.
\theoremstyle{plain}
\theoremseparator{:\quad}
\theoremprework{}
\theoremindent2em
\theoremheaderfont{\sffamily\bfseries}
\theorembodyfont{\normalfont}
\theoremsymbol{}

% redefine the over/underbrace to get more lightweight braces (borrowed from MnSymbols)
\makeatletter
\DeclareSymbolFont{largesymbolsA}{U}{txexa}{m}{n}
\def\re@DeclareMathSymbol#1#2#3#4{%
    \let#1=\undefined
    \DeclareMathSymbol{#1}{#2}{#3}{#4}}
\re@DeclareMathSymbol{\br@cext}{\mathord}{largesymbolsA}{"20}
\DeclareSymbolFont{extsymbols}{OMX}{txex}{m}{n}
\re@DeclareMathSymbol{\braceld}{\mathord}{extsymbols}{"7A}
\re@DeclareMathSymbol{\bracerd}{\mathord}{extsymbols}{"7B}
\re@DeclareMathSymbol{\bracelu}{\mathord}{extsymbols}{"7C}
\re@DeclareMathSymbol{\braceru}{\mathord}{extsymbols}{"7D}
\def\downbracefill{$\m@th%
   \braceld\mkern-1mu\cleaders\hbox{$\mkern-.5mu\br@cext\mkern-.5mu$}%
   \hfill\mkern-1mu%
   \braceru\bracelu%
   \mkern-1mu\cleaders\hbox{$\mkern-.5mu\br@cext\mkern-.5mu$}%
   \hfill\mkern-1mu\bracerd$}
 \def\upbracefill{$\m@th%
   \bracelu\mkern-1mu\cleaders\hbox{$\mkern-.5mu\br@cext\mkern-.5mu$}%
   \hfill\mkern-1mu%
   \bracerd\braceld%
   \mkern-1mu\cleaders\hbox{$\mkern-.5mu\br@cext\mkern-.5mu$}%
   \hfill\mkern-1mu\braceru$}
\makeatother

% index for multicols
\makeatletter
\renewenvironment{theindex}
  {\if@twocolumn
      \@restonecolfalse
   \else
      \@restonecoltrue
   \fi
   \begin{multicols*}{3}[\paragraph*{\indexname}]
   \markboth{\MakeUppercase\indexname}%
            {\MakeUppercase\indexname}%
   \thispagestyle{mystyle}
   \setlength{\parindent}{0pt}
   \setlength{\parskip}{0pt plus 0.3pt}
   \relax
   \let\item\@idxitem}%
  {\end{multicols*}\if@restonecol\onecolumn\else\clearpage\fi}
\makeatother
\usepackage[
	a4paper,
	landscape,
	left=.75cm,
	right=.75cm,
	top=.75cm,
	bottom=0.5cm,
	includeheadfoot
]{geometry}
%% See the TeXed file for more explanations

\usepackage{babel}

\usepackage[utf8]{inputenc}
\usepackage[sc]{mathpazo}
\usepackage[thmmarks]{ntheorem}
\usepackage[nice]{nicefrac}
\usepackage[load={prefixed,prefix,abbr,named,synchem,addn},per=frac,fraction=frac]{siunitx}
\usepackage{tikz}

%% Math related packages

%% The AMS-LaTeX extensions for mathematical typesetting.
\usepackage{amsmath,amssymb,amsfonts,mathrsfs}

\renewcommand{\arraystretch}{0.8}
% import content
\newcommand{\content}[1]{
	\input{content/#1.tex}
}
% Macro for units
\newcommand{\sunit}[1]{\ensuremath{ \quad \left[ \si{#1} \right] } }
\newcommand{\nicesunit}[1]{\ensuremath{ \quad \left[ \si[fraction=nicefrac]{#1} \right] } }
\newcommand{\unit}[1]{\ensuremath{ \left[ \si{#1} \right] } }
\newcommand{\niceunit}[1]{\ensuremath{ \left[ \si[fraction=nicefrac]{#1} \right] } }

%% [OPT] Multi-rowed cells in tabulars
%\usepackage{multirow}

%% [REC] Intelligent cross reference package. This allows for nice
%% combined references that include the reference and a hint to where
%% to look for it.
\usepackage{varioref}
% \labelformat{equation}{(#1)} % keine gute idee

%% [OPT] Easily changeable quotes with \enquote{Text}
% \usepackage[german=swiss]{csquotes}

%% [REC] Format dates and time depending on locale
% \usepackage{datetime}

%% [OPT] Provides a \cancel{} command to stroke through mathematics.
\usepackage{cancel}

%% [ADV] This allows for additional typesetting tools in
%% mathmode. \mathclap{} will allow you to take away the width of a
%% mathematical statement seen by LaTeX.
\usepackage{mathtools}

%% [ADV] Conditional commands
%% (note that this is included by macrosetup.tex anyway)
%\usepackage{ifthen}

%% [OPT] Manual large braces or other delimiters.
%\usepackage{bigdelim, bigstrut}

%% [REC] Alternate vector arrows. Use the command \vv{} to get scaled
%% vector arrows.
\usepackage[h]{esvect}

%% [NEED] Some extensions to tabulars and array environments.
\usepackage{array}

%% [OPT] Provides \unit[1]{N} as a means to facilitate unit
%% typesetting as well as \nicefrac{}{} command that prints a fraction
%% in text-height. Very useful for fractions inside matrices.
%\usepackage{units}

%% [NEED] Allows to rotate elements
\usepackage{rotating}

%% [OPT] LaTeX epic/eepic graphics format support.
%\usepackage{epic,eepic}

%% [OPT] Postscript support via pstricks graphics package. Very
%% diverse applications.
%\usepackage{pstricks,pst-all}

%% [?] This seems to allow us to define some additional counters.
%\usepackage{etex}

%% [ADV] XY-Pic to typeset some matrix-style graphics
%\usepackage[all]{xy}

\usepackage{booktabs}
% \usepackage{tabulary}

% \usepackage[caption=false,format=hang]{subfig}

% \usepackage[pdftex,bookmarksopen]{hyperref}

\makeatletter

\def\@pdfborder{0 0 0}%
\let\@pdfborderstyle\@empty

\makeatother

\usepackage{longtable}

\usepackage{etex}

\usepackage{fancybox}
\setlength\shadowsize{1pt}
\usepackage{empheq}

\usepackage[setpagesize=false]{hyperref}

\usepackage{multirow}
%% Custom commands =============================================================

%% Depends on this
\usepackage{ifthen}
\usepackage{rotating}
\usepackage{xy}

%% Differential d. This sets it as proper operator in roman type. With correct
%% spacing. ISO standards for mathematical typesetting says it should be printed
%% like this.
\newcommand{\diff}[1]{\operatorname{d}\ifthenelse{\equal{#1}{}}{\,}{\!#1}}

% Einheitsmatrix
\newcommand{\I}{\mathbb{I}}

%% Symbols for euler number and imaginary unit
\providecommand*{\eu}%
{\ensuremath{\mathrm{e}}}
% The imaginary unit
\providecommand*{\iu}%
{\ensuremath{\mathrm{i}}} % i can be replaced with j on preference.

%% Ortsvektor mit römischen buchstaben
\newcommand{\pVec}[1]{\vv{r_{#1}}}

%% Uncomment below what style you prefer for printing differential operators
%\DeclareMathOperator{\grad}{grad}
%\DeclareMathOperator{\rot}{rot}
%\DeclareMathOperator{\Div}{div}
\DeclareMathOperator{\grad}{\nabla\!}
\DeclareMathOperator{\rot}{\nabla\times}
\DeclareMathOperator{\Div}{\nabla\cdot}

% complex and real operators
\renewcommand{\Im}{\mathrm{Im}}
\renewcommand{\Re}{\mathrm{Re}}

%% Additional mathematical operators
\DeclareMathOperator{\tr}{tr}
\DeclareMathOperator{\id}{Id}
\DeclareMathOperator{\Kern}{Kern}
\DeclareMathOperator{\diag}{diag}
\DeclareMathOperator{\arccot}{arccot}
\DeclareMathOperator{\Adj}{Adj}
\DeclareMathOperator{\arsinh}{arsinh}
\DeclareMathOperator{\arcosh}{arcosh}
\DeclareMathOperator{\artanh}{artanh}
\DeclareMathOperator{\const}{const}
\DeclareMathOperator{\erf}{erf}
\DeclareMathOperator{\erfc}{erfc}
\newcommand{\transp}{\mathsf T}

\newcommand{\bVec}[1]{\mathbold{#1}}

%% German variants
%\DeclareMathOperator{\Kern}{Kern}
%\DeclareMathOperator{\Bild}{Bild}
%\DeclareMathOperator{\Grad}{Grad}
%% English variants
%% \ker is provided by LaTeX
\DeclareMathOperator{\im}{im}
%% \grad is provided by LaTeX

%% Special characters for number sets, e.g. real or complex numbers.
\newcommand{\C}{\mathbb{C}}
\newcommand{\K}{\mathbb{K}}
\newcommand{\N}{\mathbb{N}}
\newcommand{\Q}{\mathbb{Q}}
\newcommand{\R}{\mathbb{R}}
\newcommand{\Z}{\mathbb{Z}}
\newcommand{\X}{\mathbb{X}}

%% Fixed size delimiter examples
\newcommand{\floor}[1]{\lfloor #1 \rfloor}
\newcommand{\ceil}[1]{\lceil #1 \rceil}
\newcommand{\seq}[1]{\langle #1 \rangle}
\newcommand{\set}[1]{\{ #1 \}}
\newcommand{\abs}[1]{\lvert #1 \rvert}
\newcommand{\norm}[1]{\lVert #1 \rVert}
\newcommand{\indic}[1]{\bigl[#1\bigr]}

%% Scaling delimiter examples
\newcommand{\Floor}[1]{\left\lfloor #1 \right\rfloor}
\newcommand{\Ceil}[1]{\left\lceil #1 \right\rceil}
\newcommand{\Seq}[1]{\left\langle #1 \right\rangle}
\newcommand{\Set}[1]{\left\{ #1 \right\}}
\newcommand{\Abs}[1]{\left\lvert #1 \right\rvert}
\newcommand{\Norm}[1]{\left\lVert #1 \right\rVert}

%% Absolute and partial derrivate fractions
\newcommand{\Diff}[2]{\displaystyle\frac{\diff{#1}}{\diff{#2}}}
\newcommand{\Part}[2]{\displaystyle\frac{\partial #1}{\partial #2}}

%% Set an index and print it to the current position at the same time
\newcommand{\Index}[1]{\emph{#1}\index{#1}}

%% Displaystyle math for inline math mode
\newcommand{\ds}{\displaystyle}

%% Easy to use alias for the default matrices with round braces
\newcommand{\Mx}[1]{\ensuremath{\begin{bmatrix}#1\end{bmatrix}}}

%% Include a lecture from the lectures/ folder by date.
\newcommand{\Include}[4][\prefix]%
{\ifthenelse{\equal{#1}{+PRE% this comment prevents substitution
FIX+}}{\Include[TP]{#2}{#3}{#4}}{\lecturedate{\formatdate{#2}{#3}{20#4}}\input{lectures/#1-#4-#3-#2.tex}}}


%% A macro to typeset a commutitive diagram in the style of
%% \[\Abb[functionname]{from}{to}{fromelement}{toelement}\]
\newcommand{\Sidein}{\begin{rotate}{90}\small$\in$\end{rotate}}
\newcommand{\sidew}[1]{\rotatebox{90}{\small$#1$}}

\newcommand{\Abb}[5][]{\ensuremath{
    \begin{array}{lc}
      \ifthenelse{\equal{#1}{}}{}{#1:}\;\; &
      \begin{xy}
        \xymatrixrowsep{1em}\xymatrixcolsep{2em}%
        \xymatrix{ #2 \ar[r] \ar@{}[d]^<<<<{\hspace{0.001em} \Sidein}
          & #3  \ar@{}[d]^<<<<{\hspace{0.001em} \Sidein} \\
          #4 \ar@{|->}[r] & #5} \end{xy}
    \end{array}
  }%
}


%% Use the alternative epsilon per default and define the old one as \oldepsilon
\let\oldepsilon\epsilon

\renewcommand{\epsilon}{\ensuremath\varepsilon}

%% Also set the alternate phi as default.
\renewcommand{\phi}{\ensuremath{\varphi}}

%% Table centering header
\newcommand{\ctabletitle}[2]{\multicolumn{#1}{c}{\bfseries#2}}

%% emphasized equation
\newcommand{\emphequation}[2]{
	\begin{empheq}[box=\shadowbox*]{#1}
		#2
	\end{empheq}%
}

% renew \abs so the bars are adjusted to the height of the content
\renewcommand{\abs}[1]{\ensuremath{\left|#1\right|}}

% conditional equation with large left curly brace
\newcommand{\conditional}[1]{
\left\{\begin{array}{@{}l@{\quad}l}
	#1
\end{array}\right.
}

% parentheses
\newcommand{\parens}[1]{\ensuremath{\left(#1\right)}}

% 1/2 fractions
\newcommand{\half}{\ensuremath{\frac{1}{2}}}
\newcommand{\nicehalf}{\ensuremath{\nicefrac{1}{2}}}

%% tighter itemize
\newenvironment{tightitemize}
	{
	\begin{itemize}
	\setlength{\itemsep}{2pt}
	\setlength{\parskip}{0pt}
	\setlength{\parsep}{0pt}
	}
	{\end{itemize}}

%% tighter enumerate
\newenvironment{tightenumerate}
	{
	\begin{enumerate}
	\setlength{\itemsep}{2pt}
	\setlength{\parskip}{0pt}
	\setlength{\parsep}{0pt}
	}
	{\end{enumerate}}

%% Handy Envs
\newenvironment{definition}
	{\textsc{Definition:}\ }

\newenvironment{theorem}
	{\textsc{Theorem:}\ }

\newenvironment{notation}
	{\textsc{Notation:}\ }

\newenvironment{beispiel}
	{\textsc{Beispiel:}\ }

\newenvironment{satz}
	{\textsc{Satz:}\ }

\newenvironment{bedingung}
	{\textsc{Bedingung:}\ }

\newenvironment{achtung}
	{\textsc{Achtung:}\ }

\newenvironment{bedingungen}
	{
	\textsc{Bedingungen:}
	\vspace{-2ex}
	\begin{tightitemize}
	}
	{\end{tightitemize}}

\newenvironment{bemerkung}
	{\textsc{Bemerkung:}\ }
	
\newenvironment{bemerkungen}
	{
	\textsc{Bemerkungen:}
	
	\begin{tightitemize}
	}
	{\end{tightitemize}}

\newenvironment{folgerung}
	{\textsc{Folgerung:}\ }

\newenvironment{algo}
	{\textsc{Algorithmus:}\ }

\newenvironment{folgerungen}
	{
	\textsc{Folgerungen:}
	
	\begin{tightitemize}
	}
	{\end{tightitemize}}

\newenvironment{regeln}
	{
	\textsc{Regeln:}
	
	\begin{tightitemize}
	}
	{\end{tightitemize}}


\hypersetup{
    bookmarks=true,         % show bookmarks bar?
    unicode=true,          % non-Latin characters in Acrobat’s bookmarks
%    pdftoolbar=true,        % show Acrobat’s toolbar?
%    pdfmenubar=true,        % show Acrobat’s menu?
    pdffitwindow=true,     % window fit to page when opened
    pdfstartview={FitH},    % fits the width of the page to the window
    pdftitle={Lineare Algebra -- Zusammenfassung},    % title
    pdfauthor={Philipe Fatio},     % author
%    pdfsubject={Zusammenfassung},   % subject of the document
%    pdfcreator={Creator},   % creator of the document
%    pdfproducer={Producer}, % producer of the document
%    pdfkeywords={keywords}, % list of keywords
    pdfnewwindow=true,      % links in new window
    colorlinks=true,       % false: boxed links; true: colored links
    linkcolor=black,          % color of internal links
    citecolor=black,        % color of links to bibliography
    filecolor=black,      % color of file links
    urlcolor=black           % color of external links
}

\makeatletter
\renewcommand\paragraph{\@startsection{paragraph}{4}{\z@}%
  {-3.25ex\@plus -1ex \@minus -.2ex}%
  {1.5ex \@plus .2ex}%
  {\normalfont\normalsize\bfseries}}
\makeatother

\makeindex

\setlength{\parskip}{1ex plus 0.5ex minus 0.2ex}

% Theorems etc.
\newtheorem{lemma}{Lemma}[section]
\newtheorem{satz}[lemma]{Satz}
\newtheorem{korollar}[lemma]{Korollar}

% Definitions etc.
\newenvironment{definition}
	{\textsc{Definition:}\,}

\newenvironment{beispiel}
	{\textsc{Beispiel:}\,}

\newenvironment{bemerkung}
	{\textsc{Bemerkung:}\,}
	
\newenvironment{folgerung}
	{\textsc{Folgerung:}\,}

% New math functions
\DeclareMathOperator{\vspan}{span}
\DeclareMathOperator{\Bild}{Bild}
\DeclareMathOperator{\Rang}{Rang}
\DeclareMathOperator{\Spur}{Spur}

\def\Real{\mathrm{I\!R}}
\def\Complex{\mathrm{l\!\!\!C}}
\addtodef{\tableofcontents}{\pagestyle{mystyle}}{}
% HEADER (end)
\begin{document}
	\begin{multicols*}{3}
	\thispagestyle{mystyle}
	\tableofcontents
	\thispagestyle{mystyle}
	
	\section{Lineare Gleichungssysteme — Der Gauss'sche Algorithmus} % (fold)
		\subsection{Herleitung des Gaussverfahrens} % (fold)
			\begin{definition}
				Die Menge aller Lösungen eines linearen Gleichungssystem heisst \emph{Lösungsmenge}\index{Lösungsmenge} des linearen Gleichungssystems.
			\end{definition}
		
			\begin{definition}
				Der Koeffizient $a_{pj}^{(j)}$ im $j$--ten Eliminationsschema heisst \emph{Pivot}\index{Pivot} des $j$--ten Eliminationsschritts. Ein Pivot ist nach Definition immer ungleich null. Die Zeile in der das Pivot steht, heisst \emph{Pivot-Zeile}\index{Pivot!-Zeile}.
			\end{definition}
		
			\subparagraph{Gauss'scher Algorithmus:}\index{Gauss!Gauss'scher Algorithmus}  ($m$ Gleichungen, $n$ Unbekannte)
		
			Setze $i := 1$, $j := 1$.
			\begin{itemize}
				\item[$\alpha$)] Falls $i > m$ oder $j > n$ gehe zu (L); sonst
				\item[$\mathbf{(\textbf{E})}_{\boldsymbol{j}}$] Führe im $j$--ten Eliminationsschema einen Eliminationsschritt durch:
				\item[a)] Falls möglich, bestimme einen Zeilenindex $p \in \{i,\dots,m\}$, für den $a_{pj}^{(j)} \neq 0$; sonst setze $j:=j+1$ und gehe zu $\alpha)$.
			
				\item[b)] Falls $p \neq i$, vertausche diese beiden Zeilen und numeriere sie entsprechend um.
				\item[c)] Für $k=i+1,\dots,m$: Bilde $l_{ki} := a_{kj}^{(j)}/a_{ij}^{(j)}$ und subtrahiere das $l_{ki}$--fache der Zeile mit Index $k$.
				\item[$\omega$)] Setze $i:=i+1$, $j:=j+1$ und gehe zu $\alpha)$.
				\item[\textbf{(L)}] Bestimme die Lösungsmenge durch Rückwärtseinsetzen.
			\end{itemize}
		% (end)
		
		\subsection{Folgerungen aus dem Gauss'schen Endschema} % (fold)
			\begin{definition}
				Die Zahl $r$ (die Anzahl ``Nicht-Nullzeilen'' im Hauptteil des Endschemas) heisst \emph{Rang des Gleichungssystems}\index{Gleichunssystem!Rang des}. Eine über einem Pivot in der Kopfzeile des Endschemas stehende Variable heisst \emph{Pivotvariable}\index{Pivot!variable}. Variablen, die keine Pivotvariablen sind, heissen \emph{freie Parameter}\index{freie Parameter}.
			\end{definition}
			
			\begin{bemerkung}
				Es gilt $r \geq 0$ und $r \leq m$. Da die Anzahl der Pivotvariablen gleich $r$ ist, muss auch $r \leq n$ gelten. Es gibt $n - r$ freie Parameter.
			\end{bemerkung}
			
			\begin{satz}
				Ein lineares Gleichungssystem hat genau dann mindestens eine Lösung, wenn
				
				\begin{tabular}{lrl}
					entweder & \textnormal{i)} & $r=m$\\
					oder & \textnormal{ii)} & $r < m$ und $c_i = 0$, $i = r+1,\dots,m$.
				\end{tabular}

			\end{satz}
			
			\begin{satz}
				Die Lösung eines linearen Gleichungssystems --- falls sie existiert --- ist genau dann eindeutig, wenn $r = n$ ist.
			\end{satz}
			
			\begin{definition}
				Ein lineares Gleichungssystem heisst \emph{homogen}\index{Gleichungssystem!homogenes}, falls alle rechten Seiten gleich null sind.
			\end{definition}
			
			\begin{korollar}
				Ein homogenes Gleichungssystem hat genau dann nichttriviale Lösungen, wenn $r < n$ ist.
			\end{korollar}
			
			\begin{korollar}
				Ein lineares Gleichungssystem ist genau dann für beliebige rechte Seiten lösbar, wenn $r = m$ ist.
			\end{korollar}
			
			\begin{korollar}
				Ein lineares Gleichungssystem ist genau dann \emph{nicht} für beliebige rechte Seiten lösbar, wenn $r < m$ ist.
			\end{korollar}
			
			\begin{korollar}
				Sei $m = n$. Die Lösung eines linearen Gleichungssystems ist genau dann eindeutig, wenn das System für beliebige rechte Seiten lösbar ist.
			\end{korollar}
			
			\begin{korollar}
				Sei $m = n$. Ein lineares Gleichungssystem ist genau dann für beliebige rechte Seiten lösbar, wenn das zugehörige homogene System nur die triviale Lösung besitzt.
			\end{korollar}
			
			\begin{bemerkung}
				Der Rang $r$ eines linearen Gleichungssystems ist \emph{eindeutig bestimmt}, dass heisst, $r$ ist unabhängig von den vorgenommenen Zeilenvertauschungen beim Gaussverfahren.
			\end{bemerkung}
		% (end)
	% (end)
	
	\section{Matrizen} % (fold)
	
		\subsection{Definition und spezielle Matrizen} % (fold)
			\begin{definition}
				Eine \emph{$m \times n$--Matrix}\index{Matrix} ist ein Schema von $mn$ Zahlen, angeordnet in $m$ \emph{Zeilen} und $n$ \emph{Spalten}. Diese $mn$ Zahlen heissen \emph{Elemente} der $m \times n$--Matrix.
			\end{definition}
			
			\begin{beispiel}
				Das Schema
				\[
					A = \left(\begin{array}{rrr}
									2 & 3 & 1 \\
									5 & 1 & 2
								\end{array}\right)
				\]
				ist eine $2 \times 3$--Matrix. In der ersten Zeile ist das zweite Element $(A)_{12} = a_{12} = 3$.
			\end{beispiel}
			
			\begin{definition}
				Eine $n \times n$--Matrix (d.h. eine Matrix mit gleich vielen Spalten wie Zeilen) heisst \emph{quadratische Matrix}\index{Matrix!quadratische}.
			\end{definition}
			
			\begin{definition}
				\emph{Gleichheit von Matrizen}\index{Matrizen!Gleichheit von}. Zwei Matrizen $A$ und $B$ heissen \emph{gleich}, wenn die die gleiche Anzahl Zeilen und Spalten haben und die entsprechenden Elemente gleich sind:
				\[
					(A)_{ij} = (B)_{ij},\ \text{für alle } i,j
				\]
			\end{definition}
			
			\begin{definition}
				\emph{Spezielle Matrizen}\index{Matrix!spezielle}.
				
				\begin{itemize}
					\item[i)] Eine $m \times n$--Matrix heisst \emph{Nullmatrix}\index{Nullmatrix}, falls jedes Element gleich null ist. Jede Nullmatrix wird mit $0$ bezeichnet.
					
					\begin{beispiel}
						Die Matrix
						\[
							0= \left(\begin{array}{rrr}
								0 & 0 & 0 \\
								0 & 0 & 0
							\end{array}\right)
						\]
						ist die $2 \times 3$--Nullmatrix.
					\end{beispiel}
					
					\item[ii)] Eine quadratische Matrix $R$ heisst \emph{obere Dreicksmatrix}\index{Dreiecksmatrix!obere} oder \emph{Rechtsdreiecksmatrix}\index{Rechtsdreiecksmatrix}, falls $(R)_{ij} = 0$ für $i > j$.
					
					\begin{beispiel}
						\[
							R= \left(\begin{array}{rrr}
								1 & 3 & 1 \\
								0 & 2 & 4 \\
								0 & 0 & 3
							\end{array}\right).
						\]
					\end{beispiel}
					
					Eine quadratische Matrix $L$ heisst \emph{untere Dreiecksmatrix}\index{Dreiecksmatrix!untere} oder \emph{Linksdreiecksmatrix}\index{Linksdreiecksmatrix}, falls $(L)_{ij} = 0$ für $i < j$.
					
					\begin{beispiel}
						\[
							L = \left(\begin{array}{rrrr}
								2 & 0 & 0 & 0 \\
								3 & 4 & 0 & 0 \\
								1 & 2 & 2 & 0 \\
								1 & 0 & 0 & 3
							\end{array}\right).
						\]
					\end{beispiel}
					
					\item[iii)] Eine $n \times n$--Matrix $D$ heisst \emph{Diagonalmatrix}\index{Diagonalmatrix}, falls $(D)_{ij} = 0$ für $i \neq j$. Die Elemente $(D)_{ii} = d_{ii}$ heissen \emph{Diagonalelemente}\index{Diagonalelemente}. Für die Diagonalmatrix mit gegebenen Diagonalelementen $d_{11},\,d_{22},\,\dots,\,d_{nn}$ schreiben wir $D = \diag (d_{11},\,d_{22},\,\dots,\,d_{nn})$.
					
					\begin{beispiel}
						\[
							D = \left(\begin{array}{rrr}
								5 & 0 & 0 \\
								0 & -2 & 0 \\
								0 & 0 & 3
							\end{array}\right) = \diag(5,2,3).
						\]
						
					\end{beispiel}
					
					\item[iv)] Die $n \times n$--Matrix $I_n = \diag(1,1,\dots,1)$ heisst \emph{Einheitsmatrix}\index{Einheitsmatrix} oder \emph{Identität}\index{Identität}.
					
					\begin{beispiel}
						\[
							I_3 = \left(\begin{array}{rrr}
								1 & 0 & 0 \\
								0 & 1 & 0 \\
								0 & 0 & 1
							\end{array}\right).
						\]
					\end{beispiel}
					
					\item[v)] Eine weitere häufig auftretende Klasse von Matrizen sind 1-spaltige oder $n \times 1$--Matrizen. Diese $n \times 1$--Matrizen werden \emph{Spaltenvektoren}\index{Spaltenvektoren} genannt. Wir bezeichnen Spaltenvektoren mit Kleinbuchstaben. Die Elemente eines Spaltenvektors heissen \emph{Komponenten}\index{Spaltenvektor!Komponenten}. Die Komponenten eines Spaltenvektors werden nur mit dem Zeilendindex indiziert.
					
					\begin{beispiel}
						Die $4 \times 1$--Matrix $b = \left(\begin{array}{r}
							2 \\ -4 \\ 7 \\ 0
						\end{array}\right)$ ist ein Spaltenvektor.
						
						Es gilt $b = \left(\begin{array}{r}
							b_1 \\ b_2 \\ b_3 \\ b_4
						\end{array}\right)$ mit $b_1 = 2$, $b_2 = -4$, $b_3 = 7$, $b_4 = 0$.
					\end{beispiel}
				\end{itemize}
			\end{definition}
			
		% (end)
			
		\subsection{Das Rechnen mit Matrizen} % (fold)
			\begin{definition}
				\emph{Addition}\index{Matrix!Addition}. Es seien $A$ und $B$ zwei $m \times n$--Matrizen. $A$ und $B$ werden addiert, indem man entsprechende Elemente addiert. Genauer: Die $m \times n$--Matrix $A + B$ mit $(A + B)_{ij} = (A)_{ij} + (B)_{ij}$ heisst \emph{Summe} der Matrizen $A$ und $B$.
				
				\begin{beispiel}
					\[
						\left(\begin{array}{rrr}
							3 & 1 & 0 \\
							2 & -2 & 1
						\end{array}\right)
						+
						\left(\begin{array}{rrr}
							1 & 2 & 0 \\
							0 & 1 & 1
						\end{array}\right)
						=
						\left(\begin{array}{rrr}
							4 & 3 & 0 \\
							2 & -1 & 2
						\end{array}\right).
					\]
				\end{beispiel}
			\end{definition}
			
			\begin{definition}
				\emph{Multiplikation mit einer Zahl}\index{Matrix!Multiplikation mit einer Zahl}. Eine $m \times n$--Matrix $A$ wird mit einer Zahl $\alpha$ multipliziert, indem man jedes Element von $A$ mit $\alpha$ multipliziert: Die $m \times n$--Matrix $\alpha A$ mit $(\alpha A)_{ij} = \alpha (A)_{ij}$ \emph{Vielfaches} (genauer $\alpha$--faches) der Matrix $A$.
				
				\begin{beispiel}
					\[
						3 \left(\begin{array}{rrr}
							3 & 1 & 0 \\
							2 & -2 & 1
						\end{array}\right)
						=
						\left(\begin{array}{rrr}
							9 & 3 & 0 \\
							6 & -6 & 3
						\end{array}\right).
					\]
				\end{beispiel}
			\end{definition}
			
			\begin{definition}
				\emph{Multiplikation zweier Matrizen}\index{Matrix!Multiplikation mit einer Matrix}\index{Matrix!Produkt}. $A$ sei eine $m \times n$--Matrix und $B$ sei eine $n \times p$--Matrix. Die $m \times p$--Matrix $AB$, mit $(AB)_{ij} = \sum_{k=1}^n (A)_{ik} (B)_{kj}$, heisst Produkt der Matrizen $A$ und $B$.
				
				\begin{bemerkung}
					Das Produkt $AB$ kann nur gebildet werden, wenn die Anzahl Spalten von $A$ mit der Anzahl Zeilen von $B$ üereinstimmt. Das Element $(AB)_{ij}$ in der $i$--ten Zeile und der $j$--ten Spalte von $AB$ wird erhalten, wenn die $i$--te Zeile der Matrix $A$ mit der $j$--ten Spalte der Matrix $B$ ``multipliziert'' wird. Eine Zeile wird mit einer Spalte multipliziert, indem für jedes $k$ das $k$--te Element der Zeile mit dem $k$--ten Element der Spalte multipliziert wird und schliesslich alle diese Produkte addiert werden:
					\[
						(AB)_{ij} = a_{i1} b_{1j} + a_{i2} b_{2j} + \dots + a_{in} b_{nj}.
					\]
				\end{bemerkung}
				
				\begin{beispiel}
					Sei $A = \left(\begin{array}{rrr}
						3 & 1 & 0 \\
						2 & -2 & 1
					\end{array}\right)$ eine $2 \times 3$--Matrix und
					$B = \left(\begin{array}{rrrr}
						1 & 1 & 0 & 0 \\
						1 & 2 & 2 & 1 \\
						2 & -1 & -1 & 2
					\end{array}\right)$ eine $3 \times 4$--Matrix.
					
					Dann ist $AB = \left(\begin{array}{rrrr}
						4 & 5 & 2 & 1 \\
						2 & -3 & -5 & 0
					\end{array}\right)$ eine $2 \times 4$--Matrix.
				\end{beispiel}
			\end{definition}
			
			\begin{satz}
				\begin{itemize}
					\item[i)] Für $m \times n$--Matrizen $A$ und $B$ gilt das Kommutativgesetz\index{Kommutativgesetz} bezüglich der Addition:
					\[
						A + B = B + A.
					\]
					
					\item[ii)] Für $m \times n$--Matrizen $A$, $B$ und $C$ gilt das Assoziativgesetz\index{Assoziativgesetz} bezüglich der Addition:
					\[
						(A+B)+C=A+(B+C).
					\]
					
					\item[iii)] Für jede $m \times n$--Matrix $A$, $n \times p$--Matrix B und $p \times q$--Matrix $C$ gilt das Assoziativgesetz\index{Assoziativgesetz} bezüglich der Multiplikation:
					\[
						(AB)C=A(BC).
					\]
					
					\item[iv)] Für $m \times n$--Matrizen $A$, $B$ und $n \times p$--Matrizen $C$, $D$ gelten die Distributivgesetze\index{Distributivgesetz}:
					\begin{align*}
						(A+B)C&=AC+BC \\
						A(C+D)&=AC+AD.
					\end{align*}
				\end{itemize}

				\begin{bemerkung}
					\begin{enumerate}
						\item Die Matrixmultiplikation ist \emph{nicht kommutativ}, d.h. im allgemeinen gilt $AB \neq BA$.
						\item Für jede $m \times n$--Matrix $A$ gilt $I_m A = A I_n = A$. Daher kommt der Name Einheitsmatrix oder Identität für $I_m$ bzw. $I_n$.
					\end{enumerate}
				\end{bemerkung}
			\end{satz}
			
			\begin{satz}
				Sei $A = \left(a^{(1)}\ a^{(2)} \ \dots \ a^{(n)}\right)$ eine $m \times n$--Matrix mit $m$--Spaltenvektoren $a^{(j)},\, j=1,2,\dots,n$, sei $x$ ein $n$--Spaltenvektor mit Komponenten $x_1, \dots, x_n$ und sei $e^{(i)}$ der $i$--te Spaltenvektor der Einheitsmatrix $I_n$. Dann gilt:
				\begin{itemize}
					\item[i)] $Ae^{(i)} = a^{(i)} = $ $i$--ter Spaltenvektor von $A$.
					\item[ii)] $Ax = x_1 a^{(1)} + x_2 a^{(2)} + \dots + x_n a^{(n)}$.
				\end{itemize}
			\end{satz}
			
			\begin{satz}
				Sei $A$ eine $m \times n$--Matrix, $B = \left(b^{(1)} \ b^{(2)} \ \dots \ b^{(p)}\right)$ eine $n \times p$--Matrix. Dann gilt:
				\[
					AB = \left(Ab^{(1)} \ Ab^{(2)} \ \dots \ Ab^{(p)}\right),
				\]
				d.h. die $i$--te Spalte der Produktmatrix $AB$ ist das Produkt der Matrix $A$ mit dem $i$--ten Spaltenvektor $b^{(i)}$ der Matrix $B$.
			\end{satz}
			
			\begin{definition}
				Sei $A$ eine $m \times n$--Matrix. Dann heisst die $n \times m$--Matrix $A^T$, mit $\left(A^T\right)_{ij} := \left(A\right)_{ji}$ die \emph{Transponierte}\index{Matrix!Transponierte} von $A$. Eine Matrix $A$ heisst \emph{symmetrisch}\index{Matrix!symmetrische}, falls $A^T = A$ gilt.
				
				\begin{beispiel}
					\begin{enumerate}
						\item Für $A = \left(\begin{array}{rrrr}
							1 & 2 & 3 & 4 \\
							5 & 6 & 7 & 8
						\end{array}\right) \quad$
						ist
						$\quad A^T = \left(\begin{array}{rr}
							1 & 5 \\
							2 & 6 \\
							3 & 7 \\
							4 & 8
						\end{array}\right)$.
						\item Die Matrix $B = \left(\begin{array}{rrr}
							2 & 3 & -5 \\
							3 & -1 & 2 \\
							-5 & 2 & 7
						\end{array}\right) = B^T$ ist symmetrisch.
					\end{enumerate}
				\end{beispiel}
			\end{definition}
			
			Für das Transponieren gelten folgende Rechenregeln:
			
			\begin{satz}
				\begin{itemize}
					\item[i)] Für jede Matrix $A$ gilt $\left(A^T\right)^T = A$.
					\item[ii)] Für beliebige $m \times n$--Matrizen $A$ und $B$ gilt $(A+B)^T = A^T + B^T$.
					\item[iii)] Für jede $m \times n$--Matrix $A$ und jede $n \times p$--Matrix $B$ gilt $(AB)^T = B^T A^T$.
				\end{itemize}
			\end{satz}
			
			\begin{korollar}
				Seien
				\[
					A = \left(\begin{array}{c}
						a^{[1]} \\
						\vdots \\
						a^{[m]}
					\end{array}\right)
					, \qquad
					B = \left(\begin{array}{c}
						b^{[1]} \\
						\vdots \\
						b^{[n]}
					\end{array}\right)
				\]
				eine $m \times n$--Matrix mit n-Zeilenvektoren $a^{[i]}$, bzw. eine $n \times p$--Matrix mit $p$--Zeilenvektoren $b^{[j]}$, sei $y = (y_1, \dots, y_n)$ ein $n$--Zeilenvektor und sei $e^{[i]}$ der $i$--te Zeilenvektor der Einheitsmatrix $I_n$. Dann gilt:
				\begin{itemize}
					\item[i)] $e^{[i]} B = b^{[i]}$.
					\item[ii)] $yB = y_1 b^{[1]} + y_2 b^{[2]} + \dots + y_n b^{[n]}$.
					\item[iii)] $AB = \left(\begin{array}{c}
						a^{[1]} B \\
						\vdots \\
						a^{[n]} B
					\end{array}\right)$.
				\end{itemize}
			\end{korollar}
		
		% (end)
			
		\subsection{Die Inverse einer Matrix} % (fold)
			\begin{definition}
				Die $n \times n$--Matrix $X$ heisst \emph{Inverse}\index{Matrix!Inverse} der $n \times n$--Matrix $A$, falls $AX = I_n$ gilt. Falls die $n \times n$--Matrix $A$ eine Inverse hat, dann heisst die Matrix $A$ \emph{invertierbar}\index{Matrix!invertierbare} oder \emph{regulär}\index{Matrix!reguläre}, andernfalls \emph{singulär}\index{Matrix!singuläre}.
			\end{definition}
			
			\begin{lemma}
				Die Inverse\label{lemma.inverse} einer Matrix ist eindeutig bestimmt, d.h. für $n \times n$--Matrizen $A$, $X$ und $Y$ gilt: Aus $AX = I_n$ und $AY = I_n$ folgt $X=Y$.
			\end{lemma}
			
			Falls die $n \times n$--Matrix $A$ regulär ist, besitzt sie nach Lemma~\ref{lemma.inverse} eine eindeutig bestimmte Inverse. Wir bezeichnen diese mit $A^{-1}$. Im nächsten Satz stellen wir einige Eigenschaften der Inversen zusammen.
			
			\begin{satz}
				Seien $A$ und $B$ invertierbare $n \times n$--Matrizen; dann gilt:
				
				\begin{itemize}
					\item[i)] $A^{-1} A = I_n$.
					\item[ii)] $A^{-1}$ ist invertierbar und $\left(A^{-1}\right)^{-1} = A$.
					\item[iii)] $I_n$ ist invertierbar und $I_n^{-1} = I_n$.
					\item[iv)] $AB$ ist invertierbar und $(AB)^{-1} = B^{-1} A^{-1}$
					\item[v)] $A^T$ ist invertierbar und $\left(A^T\right)^{-1} = \left(A^{-1}\right)^T$.
				\end{itemize}
			\end{satz}
			
			\begin{satz}
				Für jede $n \times n$--Matrix $A$ sind folgende Aussagen äquivalent:
				
				\begin{itemize}
					\item[i)] $A$ ist invertierbar.
					\item[ii)] Das Gleichungssystem $Ax = b$ ist für jedes $b$ lösbar.
					\item[iii)] Das Gleichungssystem $Ax = 0$ hat nur die triviale Lösung $x = 0$.
				\end{itemize}
			\end{satz}
			
			\paragraph{Orthogonale Matrizen.}\index{Matrix!orthogonale}
			
			\begin{definition}
				Eine $n \times n$--Matrix $A$ heisst \emph{orthogonal}, falls $A^T A = I_n$ gilt.
			\end{definition}
			
			\begin{satz}
				Seien $A$ und $B$ orthogonale $n \times n$--Matrizen; dann gilt:
				\begin{itemize}
					\item[i)] $A$ ist invertierbar und $A^{-1} = A^T$.
					\item[ii)] $A^{-1}$ ist orthogonal.
					\item[iii)] $AB$ ist orthogonal.
					\item[iv)] $I_n$ ist orthogonal.
				\end{itemize}
			\end{satz}
		
		% (end)
			
		\subsection{LR-Zerlegung} % (fold)
			\begin{definition}
				Eine $n \times n$--Matrix, die aus der Einheitsmatrix $I_n$ durch Zeilenvertauschungen hervorgeht, heisst \emph{Permutationsmatrix}\index{Permutationsmatrix}.
			\end{definition}
			
			Führt man den Gaussalgorithmus für eine $n \times n$--Matrix $A$ durch, so lassen sich aus dem Endschema die Dreiecksmatrizen $L$ und $R$ ablesen für die $LR = PA$ gilt. Die Matrix $P$ erhält man aus der Matrix $I_n$, indem man bei dieser dieselben Zeilenvertauschungen vornimmt wie beim Gaussverfahren.
			
			Wir können nun für den allgemeinen Fall der LR-Zerlegung folgendes Resultat formulieren.
			
			\begin{satz}
				Sei $A$ eine $m \times n$--Matrix. Dann liefert das Gaussverfahren, angewandt auf die Matrix $A$, eine invertierbare $m \times m$--Linksdreiecksmatrix $L$ mit Einsen in der Diagonalen, eine $m \times n$--Matrix $R$ in Zeilenstufenform und eine $m \times m$--Permutationsmatrix P, so dass $LR =PA$ gilt. Die Matrizen $L$ (ohne Einsen), $R$ und $P$ stehen im erweiterten Endschema.
			\end{satz}
			
			\begin{bemerkung}
					$A$ ist genau dann invertierbar, wenn $R$ invertierbar ist; es gilt $A^{-1} = R^{-1} L^{-1} P$.
			\end{bemerkung}
			
			\subparagraph{LR-Algorithmus:}\index{LR-Zerlegung!LR-Algorithmus}  Lösen eines Gleichungssystems $Ax = b$ mit Hilfe der LR-Zerlegung.
		
			% TODO: Format correctly
			1) LR-Zerlegung von $A$: Bestimme mit dem Gaussverfahren Matrizen $L$, $R$ und $P$, so dass $LR = PA$. \\
			2) Vorwärtseinsetzen: Löse $Lc = Pb$ nach $c$ auf. \\
			3) Rückwärtseinsetzen: Bestimme die Lösungsmenge des Gleichungssystems $Rx = c$.
			
			\begin{definition}
				Der \emph{Rang einer Matrix}\index{Matrix!Rang} $A$ ist gleich dem Rang des linearen Gleichungssystems $Ax = 0$. Er wird mit Rang $A$ bezeichnet.
			\end{definition}
		% (end)
	
	% (end)
	
	\section{Determinanten} % (fold)
		Die Determinante einer quadratischen Matrix chrarakterisiert, ob eine Matrix regulär (Determinante $\neq 0$) oder singulär (Determinante $= 0$) ist.
		
		\subsection{Definition und Eigenschaften} % (fold)
			Wir ordnen jeder quadratischen Matrix eine Zahl zu, die Determinante. Diese Zahl bezeichnen wir mit $\det A$ oder $|A|$.
			
			\begin{definition}
				\begin{enumerate}
					\item Für $1 \times 1$--Matrizen $A = (a)$ ist
					\[
						\det A = |A| = a.
					\]
					\item Sei
					\[
						A = \left(\begin{array}{cccc}
									a_{11} & a_{12} & \cdots & a_{1n} \\
									a_{21} & a_{22} & \cdots & a_{2n} \\
									\vdots & & & \vdots \\
									a_{n1} & a_{n2} & \cdots & a_{nn}
							\end{array}\right)
					\]
					eine $n \times n$--Matrix mit $n \geq 2$. Für $i = 1,2,\dots,n$ sei $A_{i1}$ die $(n-1) \times (n-1)$--Matrix, die man erhält, wenn bei $A$ die $i$--te Zeile und die erste Spalte gestrichen werden. Dann heisst diese Zahl
					\[
						\det A = |A| := a_{11} \det A_{11} - a_{21} \det A_{21} + a_{31} \det A_{31} - \cdots + (-1)^{n+1} a_{n1} \det A_{n1}
					\]
					\emph{Determinante}\index{Determinante} von A.
				\end{enumerate}
			\end{definition}
			
			\begin{beispiel}
				\begin{enumerate}
					\item \[
						\left|\begin{array}{rr}
									3&2\\
									1&2
							\end{array}\right|
						= 3 \cdot 2 - 1 \cdot 2 = 4.
					\]
					\item \begin{align*}
						\left|\begin{array}{rrr}
									1&2&1\\
									2&3&2\\
									4&1&2
							\end{array}\right|
						&= 1
						\left|\begin{array}{rr}
									3&2\\
									1&2
							\end{array}\right|
						- 2
						\left|\begin{array}{rr}
									2&1\\
									1&2
							\end{array}\right|
						+ 4
						\left|\begin{array}{rr}
									2&1\\
									3&2
							\end{array}\right| \\
						&= 1 \cdot 4 - 2 \cdot 3 + 4 \cdot 1 = 2.
					\end{align*}
				\end{enumerate}
			\end{beispiel}
			
			\begin{satz}
				\begin{itemize}
					\item[ii)] Werden in der Matrix $A$ zwei Zeilen vertauscht, so ändert die Determinante ihr Vorzeichen.
					\item[ii)] Wird in der Matrix $A$ zu einer Zeile ein Vielfaches einer anderen Zeile addiert, so bleibt die Determinante unverändert.
					\item[iii)] Wird in der Matrix $A$ eine Zeile mit einem Faktor $\alpha$ multipliziert, dann vervielfacht sich die Determinante um den Faktor $\alpha$.
				\end{itemize}
			\end{satz}
			
			\begin{folgerung}
				\begin{enumerate}
					\item Die Determinante einer Matrix mit zwei gleichen Zeilen ist gleich null.
					\item Die Determinante einer Matrix, die eine Zeile aus lauter Nullen enthält, ist gleich null.
				\end{enumerate}
			\end{folgerung}
			
			\begin{lemma}
				Die Determinante einer Dreiecksmatrix ist gleich dem Produkt ihrer Diagonalelemente.
			\end{lemma}
			
			\begin{satz}
				Für jede $n \times n$--Matrix $A$ gilt:
				\[
					\det A^T = \det A.
				\]
			\end{satz}
			
			\begin{lemma}
				Für jede $n \times n$--Matrix $A$, $n \geq 2$, gilt
				\[
					\det A = \sum_{j=1}^n (-1)^{j+1} a_{1j} \det A_{1j}.
				\]
				Dabei bezeichnet $A_{1j}$ diejenige $(n-1) \times (n-1)$--Matrix, die entsteht, wenn bei $A$ die erste Zeile und die $j$--te Spalte gestrichen werden.
			\end{lemma}
			
			\begin{korollar}
				\begin{itemize}
					\item[i)] Werden in der Matrix $A$ zwei spalten vertauscht, so ändert die Determinante ihr Vorzeichen.
					\item[ii)] Wird in der Matrix $A$ zu einer Spalte ein Vielfaches einer anderen Spalte addiert, so bleibt die Determinante unverändert.
					\item[iii)] Wird in der Matrix $A$ eine Spalte mit einem Faktor $\alpha$ multipliziert, dann vervielfacht dich die Determinante um den Faktor $\alpha$.
				\end{itemize}
			\end{korollar}
			
			\begin{satz}
				Für beliebige $m \times m$--Matrizen $A$ und $B$ gilt:
				\[
					\det AB = (\det A)(\det B).
				\]
			\end{satz}
			
			\begin{lemma}
				Sei $A$ eine $m \times m$--Matrix, $B$ eine $m \times n$--Matrix, $C$ eine $n \times n$--Matrix und sei $M$ die durch
				\[
					\left(\begin{array}{rr}
								A&B\\
								0&C
						\end{array}\right)
				\]
				gegebene $(m + n) \times (m + n)$--Matrix. Dann gilt
				\[
					\det M = (\det A)(\det C).
				\]
			\end{lemma}
			
			\begin{korollar}
				Ist die $n \times n$--Matrix $A$ invertierbar, so gilt $\det A \neq 0$ und
				\[
					\det A^{-1} = \frac{1}{\det A}
				\]
			\end{korollar}
			
		% (end)
		
		\subsection{Die effiziente Berechnung von Determinanten} % (fold)
			\begin{satz}
				Sei eine $m \times m$--Matrix $A$ gegeben. Wird der Gaussalgorithmus auf $A$ angewendet, so liefert er Matrizen $L$, $R$, und $P$ mit $LR = PA$ und es gilt
				\[
					\det A = (\det P)(\det R) = (-1)^{\text{Anzahl Zeilenvertauschungen}} \det R,
				\]
				wobei $\det R = r_{11} r_{22} \dots r_{nn}$ ist.
			\end{satz}
			
		% (end)
		
		\subsection{Determinanten und lineare Gleichnungssysteme} % (fold)
			
			\begin{lemma}
				Sei $A$ eine $n \times n$--Matrix. Wenden wir auf $A$ das Gauss'sche Eliminationsverfahren an, so ist $\det A \neq 0$ genau dann, wenn im Endschema $r = n$ gilt.
			\end{lemma}
			
			\begin{satz}
				Für jede $n \times n$--Matrix sind folgende Aussagen äquivalent:
				\begin{itemize}
					\item[i)] Die Matrix $A$ ist invertierbar.
					\item[ii)] $\det A \neq 0$.
					\item[iii)] Im Gauss--Endschema ist $r = n$.
					\item[iv)] Das lineare Gleichnungssystem $Ax = b$ ist für jedes $b$ lösbar.
					\item[v)] Die Lösung des linearen Gleichnungssystems $Ax = b$ ist eindeutig bestimmt.
					\item[vi)] Das lineare Gleichungssystem $Ax = 0$ hat nur die triviale Lösung $x = 0$.
				\end{itemize}
			\end{satz}
			
			\begin{korollar}
				Sei $A$ eine $n \times n$--Matrix. Dann gilt:
				\begin{itemize}
					\item[i)] Das homogene lineare Gleichungssystem $Ax = 0$ hat genau dann nur die triviale Lösung, wenn $\det A \neq 0$.
					\item[ii)] Das lineare Gleichungssystem $Ax = b$ ist genau dann für beliebige rechte Seiten lösbar, wenn $\det A \neq 0$ ist.
					\item[iii)] Die Lösung des linearen Gleichungssystems $Ax = b$ ist genau eindeutig, wenn $\det A \neq 0$ gilt.
				\end{itemize}
			\end{korollar}
			
			Die wichtigsten Aussagen über die Lösungsmenge eines linearen Gleichungssystems von $n$ Gleichungen und $n$ Unbekannten in Abhängigkeit der Determinante der Koeffizientenmatrix $A$:
			
			\begin{itemize}
				\item Falls $|A| \neq 0$, hat das \emph{homogene} lineare Gleichungssystem $Ax = 0$ \emph{nur} die triviale Lösung.
				\item Falls $|A| = 0$, hat das \emph{homogene} lineare Gleichungssystem $Ax = 0$ \emph{unendlich viele} Lösungen.
				\item Falls $|A| \neq 0$ hat das lineare Gleichungssystem $Ax = b$ für beliebige rechte Seiten $b$ \emph{genau eine} Lösung.
				\item Falls $|A| = 0$, hat das lineare Gleichungssystem $Ax = b$ --- abhängig von der rechten Seite $b$ --- \emph{keine oder unendlich viele} Lösungen.
			\end{itemize}
		% (end)

	% (end)
	
	\section{Vektorräume} % (fold)
		\subsection{Definition und Beispiele} % (fold)
			\begin{definition}
				Ein \emph{reeller Vektorraum} $V$ ist eine Menge con Objekten (Vektoren) mit den folgenden Eigenschaften: \\
				Es ist eine \emph{Addition} definiert, d.h. zwei Vektoren $a$, $b$ aus $V$ ist ein dritter Vektor aus $V$ zugeordnet, der mit $a + b$ bezeichnet wird. Es ist eine \emph{Multiplikation mit reellen Zahlen} definiert, d.h. einer reellen Zahl $\alpha$ und einem Vektor $a$ aus $V$ ist ein Vektor aus $V$ zugeordnet, der mit $\alpha a$ bezeichnet wird.
			\end{definition}
		% (end)
		
		\subsection{Die Struktur von Vektorräumen} % (fold)
			\begin{definition}
				Eine nichtleere Teilmenge $U$ eines Vektorraumes $V$ heisst \emph{Unterraum} von $V$, falls folgende zwei Bedingungen erfüllt sind:
				\begin{itemize}
					\item[(a)] Mit $a,\ b \in U$ ist auch $a + b \in U$.
					\item[(b)] Mit $a \in U$, $\alpha$ eine Zahl, ist auch $\alpha a \in U$.
				\end{itemize}
			\end{definition}
			
			\begin{bemerkung}
				Es gibt immer die folgenden trivialen Unterräume des Vektorraumes $V$: $V$ selbst und ${0}$, d.h. die Menge, die nur aus dem Nullvektor besteht.
			\end{bemerkung}
			
			\begin{definition}
				Kann jeder Vektor $b$ einer Vektorraumes $V$ als Linearkombination der Vektoren $a^{(1)}, a^{(2)}, \dots, a^{(k)}$ von $V$ dargestellt werden, so nennt man diese Vektoren ein \emph{Erzeugendensystem} des Vektorraumes $V$.\\
				$U$ heisst der \emph{von} $a^{(1)}, a^{(2)}, \dots, a^{(k)}$ \emph{aufgespannte} oder \emph{erzeugte Unterraum}. Man deutet das durch folgende Schreibweise an: \[
					U = \vspan \left\{a^{(1)}, a^{(2)}, \dots, a^{(k)}\right\}
				\]
			\end{definition}
			
			\begin{bemerkung}
				Die Vektoren eines Erzeugendensystems eines Vektorraumes $V$ nennt man auch \emph{erzeugend}. Man beachte, dass wir hier nur \emph{endliche} Erzeugendensysteme betrachten.
			\end{bemerkung}
			
			\begin{definition}
				Ein Vektorraum $V$ heisst \emph{endlichdimensional}, wenn er ein Erzeugendensystem besitzt, d.h. wenn es Vektoren $a^{(1)}, a^{(2)}, \dots, a^{(k)}$ gibt, so dass $V = \vspan \left\{a^{(1)}, a^{(2)}, \dots, a^{(k)}\right\}$ gilt. Andernfalls heisst der Vektorraum \emph{unendlichdimensional}.
			\end{definition}
			
			\begin{definition}
				\begin{itemize}
					\item[i)] Die Vektoren $a^{(1)}, a^{(2)}, \dots, a^{(k)}$ aus dem Vektorraum $V$ heissen \emph{linear unabhängig}, falls aus \[
						x_1 a^{(1)} + x_2 a^{(2)} + \dots + x_k a^{(k)} = 0
					\] folgt, dass $x_1 = x_2 = \dots = x_k = 0$ gilt.
					\item[ii)] Die Vektoren $a^{(1)}, a^{(2)}, \dots, a^{(k)}$ aus dem Vektorraum $V$ heissen \emph{linear abhängig}, falls es Zahlen $x_1, x_2, \dots, x_k$ gibt, welche nicht alle null sind und für die \[
						x_1 a^{(1)} + x_2 a^{(2)} + \dots + x_k a^{(k)} = 0
					\] gilt.
				\end{itemize}
			\end{definition}
			
			\begin{definition}
				Sei $a^{(1)}, a^{(2)}, \dots, a^{(k)}$ ein Erzeugendensystem für einen Vektorraum $V$. Falls die Vektoren $a^{(1)}, a^{(2)}, \dots, a^{(k)}$ linear unabhängig sind, nennt man das Erzeugendensystem eine \emph{Basis} für $V$.
			\end{definition}
			
			\begin{satz}
				Verschiedene Basen für einen Vektorraum bestehen aus gleich vielen Vektoren.
			\end{satz}
			
			\begin{lemma}
				Es sei $V$ ein Vektorraum. Die Vektoren $a^{(1)}, a^{(2)}, \dots, a^{(k)}$ aus $V$ seien linear unabhängig und die Vektoren $b^{(1)}, b^{(2)}, \dots, b^{(l)}$ seien ein Erzeugendensystem für $V$. Dann gilt $k \leq l$.
			\end{lemma}
			
			\begin{definition}
				Besitzt der Vektorraum $V \neq {0}$ eine Basis $b^{(1)}, b^{(2)}, \dots, b^{(n)}$, so heisst $n$ die \emph{Dimension von $V$} und man schreibt $n = \dim V$.
			\end{definition}
			
			\begin{bemerkung}
				\begin{enumerate}
					\item Der Nullvektor eines Vektorraumes ist immer linear abhängig, da die Gleichung $x_1 0 = 0$ eine nichttriviale Lösung besitzt, z.B. $x_1 = 1$. Der triviale Vektorraum ${0}$ hat also keine Basis. Man setzt daher $\dim \{0\} = 0$.
					\item Ist $V$ ein unendlichdimensionaler Vektorraum, so setzt man $\dim V = \infty$.
				\end{enumerate}
			\end{bemerkung}
			
			\begin{satz}
				Es sei $V$ ein Vektorraum der Dimension $n$. Dann gilt:
				\begin{itemize}
					\item[i)] Mehr als $n$ Vektoren in $V$ sind linear abhängig.
					\item[ii)] Weniger als $n$ Vektoren in $V$ sind nicht erzeugend.
					\item[iii)] $n$ Vektoren in $V$ sind linear unabhängig genau dann, wenn sie erzeugend sind, und genau dann bilden sie eine Basis.
				\end{itemize}
			\end{satz}
			
			\begin{definition}
				Die Zahlen $x_1, x_2, \dots, x_n$ in $x = \sum^n_{i=1} x_i a^{(i)}$ heissen \emph{Koordinaten} des Vektors $x$ bezüglich der Basis $a^{(1)}, a^{(2)}, \dots, a^{(n)}$.
			\end{definition}
			
			\begin{bemerkung}
				Der Rang einer Matrix $A$ ist gleich der maximalen Anzahl linear unabhängiger Spaltenvektoren von $A$.
			\end{bemerkung}
			
		% (end)
		
		\subsection{Normierte Vektorräume} % (fold)
			\begin{definition}
				Sei $V$ ein Vektorraum. Eine Vorschrift, die jedem Vektor $a \in V$ eine reelle Zahl $||a||$ zuordnet, heisst \emph{Norm} (oder \emph{Länge}) in $V$, falls die folgende Regeln erfüllt sind:
				\begin{itemize}
					\item[(N1)]
						\begin{itemize}
							\item[i)] Für jeden Vektor $a \in V$ gilt $||a|| \geq 0$,
							\item[ii)] aus $||a|| = 0$ folgt $a = 0$.
						\end{itemize}
					\item[(N2)] Für jeden Vektor $a \in V$ und für jede Zahl $\alpha$ gilt: $||\alpha a|| = |\alpha| ||a||$.
					\item[(N3)] Für alle Vektoren $a, b \in V$ gilt: $||a+b|| \leq ||a|| + ||b||$ (Dreiecksungleichung).
				\end{itemize}
			\end{definition}
			
			\begin{satz}
				Sei $V$ ein endlichdimensionaler Vektorraum, und seien $||x||$ und $||x||'$, $x \in V$, zwei Normen. Dann gibt es eine Konstante $c \geq 1$, so dass für jeden Vektor $x \in V$ gilt: \[
					\frac{1}{c} ||x||' \leq ||x|| \leq c ||x||'.
				\]
			\end{satz}
			
		% (end)
		
		\subsection{Das Skalarprodukt} % (fold)
			\begin{definition}
				Sei $V$ ein reeller Vektorraum. Eine Vorschrift, die jedem Paar $x$, $y$ von Vektoren aus $V$ eine reelle Zahl $(x, y)$ zuordnet, heisst \emph{Skalarprodukt im Vektorraum $V$}, falls gilt:
				\begin{itemize}
					\item[(S1)] Das Skalarprodukt ist \emph{linear im zweiten Faktor}, d.h. es gilt
						\begin{itemize}
							\item[i)] $(x, y^{(1)} + y^{(2)}) = (x, y^{(1)}) + (x, y^{(2)})$ für alle $x,\ y^{(1)},\ y^{(2)} \in V$;
							\item[ii)] $(x, \alpha y) = \alpha(x, y)$ für alle $x,\ y \in V,\ \alpha \in \Real$;
						\end{itemize}
					\item[(S2)] Das Skalarprodukt ist \emph{symmetrisch}, d.h. es gilt \\
						$(x, y) = (y, x)$ für alle $x,\ y \in V$.
					\item[(S3)] Das Skalarprodukt ist \emph{positiv definit}, d.h. es gilt
						\begin{itemize}
							\item[i)] $(x, x) \geq 0$ für alle $x \in V$;
							\item[ii)] aus $(x, x) = 0$ folgt $x = 0$.
						\end{itemize}
				\end{itemize}
			\end{definition}
			
			\begin{definition}
				Zwei Vektoren stehen \emph{orthogonal} (man sagt auch, sie \emph{stehen senkrecht aufeinander}), falls $(x, y) = 0$ ist.
			\end{definition}
			
			\begin{satz}
				Sei $V$ ein reeller Vektorraum mit Skalarprodukt.
				\begin{itemize}
					\item[i)] Die orthogonale Projektion eines Vektors $x$ auf den Vektor $y \neq 0$ ist gegeben durch den Vektor $\frac{(y,x)}{(y,y)}y$.
					\item[ii)] Für alle $x,\ y \in V$ gilt \[
						(x, y)^2 \leq (x, x) (y, y) \quad \textit{(Schwarzsche Ungleichung)}.
					\]
					\item[iii)] Die Vorschrift, die jedem $x \in V$ die Zahl \[
						||x|| := \sqrt{(x,x)}
					\] zuordnet, ist eine Norm in $V$.
					\item[iv)] Stehen zwei Vektoren $x,\ y \in V$ senkrecht aufeinander, d.h. ist $(x,y) = 0$, so gilt \[
						||x + y||^2 = ||x - y||^2 = ||x||^2 + ||y||^2 \quad \textit{(Satz von Pythagoras)}.
					\] Dabei bezeichnet $||x||,\ x \in V$ die in iii) eingeführte Norm.
				\end{itemize}
			\end{satz}
			
			\begin{bemerkung}
				\begin{enumerate}
					\item $||x|| := \sqrt{(x,x)}$ heisst die \emph{vom Skalarprodukt} $(x,y)$ \emph{induzierte Norm}. Wir vereinbaren: Falls im Vektorraum V ein Skalarprodukt definiert ist, so bezeichnet $||\cdot||$ die von diesem Skalarprodukt induzierte Norm.
					\item Nicht jede Norm ist von einem Skalarprodukt induziert, z.B. gibt es kein Skalarprodukt in $\Real^n$, das die Maximumnorm induziert.
				\end{enumerate}
			\end{bemerkung}
			
			\begin{definition}
				Ein Vektor $x$ der Länge $||x|| = 1$ heisst \emph{Einheitsvektor}.
			\end{definition}
			
			\begin{satz}
				Seien $e^{(1)}, e^{(2)}, \dots, e^{(k)}$ paarweise orthogonale Einheitsvektoren in einem reellen $n$-dimensionalen Vektorraum. Dann gilt: Die $k$ Vektoren $e^{(1)}, e^{(2)}, \dots, e^{(k)}$ sind linear unabhängig.
			\end{satz}
			
			\begin{korollar}
				In einem reellen $n$-dimensionalen Vektorraum bilden $n$ paarweise orthogonale Einheitsvektoren eine Basis.
			\end{korollar}
			
			\begin{definition}
				Eine Basis aus paarweise orthogonalen Einheitsvektoren heisst \emph{orthonormale Basis}.
			\end{definition}
			
			\begin{satz}
				In einem reellen $n$-dimensionalen Vektorraum $V$ sei eine Basis $b^{(1)}, b^{(2)}, \dots, b^{(n)}$ gegeben. Diese Basis lässt sich zu einer orthonormalen Basis $e^{(1)}, e^{(2)}, \dots, e^{(n)}$ umformen mit folgender Eigenschaft:\\
				\hspace*{1cm} $e^{(1)}$ spannt den gleichen Unterraum auf wie $b^{(1)}$;\\
				\hspace*{1cm} $e^{(1)},\ e^{(2)}$ spannen den gleichen Unterraum auf wie $b^{(1)},\ b^{(2)}$;\\
				\hspace*{1cm} $e^{(1)},\ e^{(2)},\ e^{(3)}$ spannen den gleichen Unterraum auf wie $b^{(1)},\ b^{(2)},\ b^{(3)}$;\\
				usw.
			\end{satz}
			
			\paragraph{Schmidtsches Orthogonalisierungsverfahren} % (fold)
				\begin{description}
					\item[\emph{1. Schritt:}] Setze $e^{(1)} := \frac{1}{||b^{(1)}||} b^{(1)}$.
					\item[\emph{2. Schritt:}]
						\begin{itemize}
							\item[a)] Orthogonale Projektion von $b^{(2)}$ auf $e^{(1)}$:\\ \[
								c^{(2)} := b^{(2)} - (b^{(2)}, e^{(1)}) e^{(1)}.
							\]
							\item[b)] $c^{(2)}$ normieren zu $1$: \[
								e^{(2)} := \frac{1}{||c^{(2)}||} c^{(2)}.
							\]
						\end{itemize}
						
					\item[\emph{Schritt 3:}]
						\begin{itemize}
							\item[a)] Projiziere den Vektor $b^{(3)}$ auf $e^{(1)}$ und $e^{(2)}$ und definiere dann \[
								c^{(3)} := b^{(3)} - (b^{(3)}, e^{(1)}) e^{(1)} - (b^{(3)}, e^{(2)}) e^{(2)}.
							\]
							\item[b)] Normiere $c^{(3)}$ auf 1: \[
								e^{(3)} := \frac{1}{||c^{(3)}||} c^{(3)}.
							\]
						\end{itemize}
				\end{description}
				So wird Schritt für Schritt die neue Basis $e^{(1)},e^{(2)},\dots,e^{(n)}$ konstruiert.
			% (end)
			
		% (end)
		
	% (end)
	
	\section{Ausgleichsrechnung — Methode der kleinsten Quadrate} % (fold)
		\subsection{Die Methode der kleinsten Quadrate --- Normalgleichung} % (fold)
			\begin{equation*}
				\begin{array}{ccccccc}
					x_1 & & & - & 31 & = & r_1 \\
					 & & x_2 & - & 62 & = & r_2 \\
					x_1 & + & x_2 & - & 90 & = & r_3
				\end{array}
			\end{equation*}
			
			\begin{equation}
				Ax - \mathbf{c} = \mathbf{r} \label{eq.qr1}
			\end{equation}
			
			\begin{equation}
				A^T Ax = A^T \mathbf{c} \label{eq.qr2}
			\end{equation}
			
			\begin{satz}
				\begin{itemize}
					\item[i)] Ist $x*$ Lösung der Normalgleichungen \eqref{eq.qr2}, so minimiert $x*$ die Fehlergleichung \eqref{eq.qr1} im Sinne der kleinsten Quadrate.
					\item[ii)] Sind die Spalten der Koeffizientenmatrix $A$ der Fehlergleichung \eqref{eq.qr1} linear unabhängig, so besitzen die Normalengleichungen \eqref{eq.qr2} eine eindeutig bestimmte Lösung.
				\end{itemize}
			\end{satz}
		% subsection: Die Methode der kleinsten Quadrate --- Normalgleichung (end)
		\subsection{Die Methode der kleinsten Quadrate --- QR-Zerlegung} % (fold)
			\begin{satz}
				\begin{itemize}
					\item[i)] Zu jeder $m \times n$-Matrix $A$, mit $m \geq n$, existiert eine orthogonale $m \times m$-Matrix $Q$, so dass gilt
					\[
						 A = QR \quad \text{mit} \quad R = \left(\frac{R_0}{0}\right) \ ,
					\]
					wobei $R_0$ eine $n \times n$-Rechtsdreiecksmatrix ist und $0$ die $(m-n) \times n$-Nullmatrix.
					\item[ii)] Sind die Spaltenvektoren $\mathbf{a}^{(1)},\dots,\mathbf{a}^{(n)}$ der Matrix $A$ linear unabhängig, so ist die Matrix $R_0$ regulär.
				\end{itemize}
			\end{satz}
			
			\paragraph{Algorithmus:} % (fold)
				Lösen der Fehlergleichung $Ax - \mathbf{c} = \mathbf{r}$ mit Hilfe der QR-Zerlegung:
				\[
					\begin{array}{r@{\quad}r@{\quad}ll}
						1) & R:= & Q^T A & \text{(QR-Zerlegung von $A$ mit} \\
						&&& \text{ Givens-Rotationen)} \\[5pt]
						2) & \mathbf{d}:= & Q^T \mathbf{c} & \text{(Transformation von $\mathbf{c}$)} \\[5pt]
						3) & R_0 x = & d_0 & \text{(Rückwärtseinsetzen)}
					\end{array}
				\]
			% paragraph: Algorithmus: (end)
		% subsection: Die Methode der kleinsten Quadrate --- QR-Zerlegung (end)
	% (end)
	
	\section{Lineare Abbildungen} % (fold)
		\begin{definition}
			Eine Abbildung $\mathcal{F}\ : \quad x \in V \longmapsto y = \mathcal{F}(x) \in W$ heisst \emph{lineare Abbildung vom endlichdimensionalen Vektorraum $V$ in den endlichdimensionalen Vektorraum $W$}, falls
			\begin{itemize}
				\item[i)] $\mathcal{F}(x + y) = \mathcal{F}(x) + \mathcal{F}(y) \qquad$ für alle $x,\ y \in V$ \\
				\item[ii)] $\mathcal{F}(\alpha x) = \alpha \mathcal{F}(x) \qquad$ für alle Zahlen $\alpha$ und alle $x \in V$
				gilt.
			\end{itemize}
		\end{definition}
		
		Jede beliebige lineare Abbildung $\mathcal{F}$ lässt sich durch eine $m \times n$-Matrix $A$ beschreiben.
		
		\subsection{Lineare Abbildungen und Matrizen} % (fold)
			\begin{definition}
				Sei $\mathcal{F}\::\ x \in V^n \longmapsto y = Ax \in V^m$ eine lineare Abbildung.
				\begin{itemize}
					\item[i)] Die Menge aller Vektoren, welche auf null abgebildet werden, heisst \emph{Kern der Matrix $A$}.
					\[
						\Kern A := \{x \in V^n | \: Ax = 0\}
					\]
					\item[ii)] Die Menge aller Bildvektoren $y \in V^m$ heisst \emph{Bild der Matrix $A$}.
					\begin{align*}
						\Bild A := \{ & y \in V^m | \text{ Es gibt ein $x \in V^n$,} \\ & \text{so dass $y = Ax$} \}
					\end{align*}
				\end{itemize}
			\end{definition}
			
			\begin{satz}
				Sei $A = (a^{(1)} \ \dots \ a^{(n)})$ eine $m \times n$-Matrix. Dann gilt
				\begin{itemize}
					\item[i)] $b$ liegt genau dann im Bild von $A$, wenn das Gleichungssystem $Ax = b$ lösbar ist.
					\[
						\Bild A = \vspan \left\{ a^{(1)}, a^{(2)}, \dots , a^{(n)} \right\}
					\]
					\item[ii)] Der Kern von $A$ ist die Lösungsmenge des homogenen Gleichungssystems $Ax = 0$.
					\item[iii)] $\Kern A$ ist ein Unterraum von $V^n$. \\
					$\Bild A$ ist ein Unterraum von $V^m$.
					\item[iv)] $\dim(\Kern A) + \dim(\Bild A) = n = \dim V^n$.
					\item[v)] $\dim(\Bild A) = \dim(\Bild A^T)$
				\end{itemize}
			\end{satz}
			
			\begin{korollar}
				Sei $A$ eine $m \times n$-Matrix, $B_1$ eine reguläre $m \times m$-Matrix und $B_2$ eine reguläre $n \times n$-Matrix. Dann gilt
				\begin{itemize}
					\item[i)] $\Rang A = \Rang A^T$.
					\item[ii)] $\Rang B_1 A = \Rang A$.
					\item[iii)] $\Rang A B_2 = \Rang A$.
				\end{itemize}
			\end{korollar}
			
			\begin{satz}
				Sei $V = \Real^n$ oder $V = \Complex^n$ und seien
				$b^{(1)}, \dots, b^{(k)} \in V,\ k < n$, linear unabhängig.
				Dann gibt es Vektoren $b^{(k+1)}, \dots ,b^{(n)} \in V$, so dass
				$b^{(1)}, \dots, b^{(n)}$ eine Basis in $V$ bilden.
			\end{satz}
			
			\paragraph{Das Zusammensetzen von Abbildungen:} % (fold)
				\[
					\mathcal{H}(x) = \mathcal{G}(\mathcal{F}(x)) = B(Ax) = (BA)x
				\]
				
				\begin{satz}
					\begin{itemize}
						\item[i)] Die Zusammensetzung von linearen Abbildungen ist wiederum linear.
						\item[ii)] Für $\mathcal{F}: x \in V^n \longmapsto y = Ax \in V^m$ und $\mathcal{G}: y \in V^m \longmapsto z = By \in V^p$ ist die zusammengesetzte Abbildung $\mathcal{H} := \mathcal{G} \circ \mathcal{F}$ gegeben durch $\mathcal{H} : x \in V^n \longmapsto z = BAx \in V^p$.
					\end{itemize}
				\end{satz}
			% paragraph: Das Zusammensetzen con Abbildungen: (end)
		% subsection: Lineare Abbildungen und Matrizen (end)
		\subsection{Lineare Abbildungen und Skalarprodukt} % (fold)
			\begin{satz}
				\begin{itemize}
					\item[i)] Die Unterräume $\Bild A$ und $\Kern A^T$ von $\Real^m$ spannen $\Real^m$ auf:
					\[
						\Bild A + \Kern A^T = \Real^m.
					\]
					\item[ii)] Die Unteräume $\Bild A$ und $\Kern A^T$ von $\Real^m$ stehen senkrecht aufeinander:
					Aus $y \in \Bild A$ und $v \in \Kern A^T$ folgt $(y,v) = y^T v = 0$.
					\item[iii)] $\dim (\Kern A) + \dim (\Kern A^T) = \dim \Real^m = m$
				\end{itemize}
			\end{satz}
			
			\begin{korollar}
				Das Gleichungssystem $Ax = b$ ist genau dann lösbar, wenn $b$ senkrecht auf allen Lösungen des sogenannten \emph{adjungierten} Gleichungssystems $A^T y = 0$ steht.
			\end{korollar}
		% subsection: Lineare Abbildungen und Skalarprodukt (end)
		\subsection{Lineare Selbstabbildungen von Vektorräumen} % (fold)
			\begin{definition}
				\begin{itemize}
					\item[i)] Eine Abbildung $\mathcal{F}: x \in V^n \longmapsto x' \in V^n$ heisst \emph{umkehrbar} oder \emph{invertierbar}, falls es zu jedem $x' \in V^n$ ein eindeutig bestimmtes $x \in V^n$ gibt mit $\mathcal{F}(x) = x'$.
					\item[ii)] Ist $\mathcal{F}$ invertierbar, so heisst die Abbildung, die jedem $x' = \mathcal{F}(x)$ das eindeutig bestimmte Urbild $x$ zuordnet, die \emph{Umkehrabbildung} von $\mathcal{F}$. Diese wird mit $\mathcal{F}^{-1}$ bezeichnet.
				\end{itemize}
			\end{definition}
			
			\begin{satz}
				\begin{itemize}
					\item[i)] Eine lineare Abbildung $\mathcal{F} : x \in V^n \longmapsto x' = Ax \in V^n$ ist genau dann umkehrbar, wenn $A$ regulär ist.
					\item[ii)] Ist $\mathcal{F} : x \longmapsto x' = Ax$ umkehrbar, so ist $\mathcal{F}^{-1}$ linear und $\mathcal{F}^{-1}$ wird durch die Matrix $A^{-1}$ beschrieben. $\mathcal{F}^{-1}: x' \longmapsto x = A^{-1} x'$.
					\item[iii)] Ist $\mathcal{F}$ umkehrbar, so gilt $\mathcal{F}^{-1} \circ \mathcal{F} = \mathcal{F} \circ \mathcal{F}^{-1} = \mathcal{I}$. Dabei ist $\mathcal{I}$ die Identität, d.h. $\mathcal{I} : x \in V^n \longmapsto x \in V^n$.
				\end{itemize}
			\end{satz}
			
			\paragraph{Koordinatentransformation} % (fold)
			
				\[
					x = Ty
				\]
			
				\begin{satz}
					Seien eine lineare Abbildung $\mathcal{F} : x \in V^n \longmapsto x' = Ax \in V^n$ und eine Koordinatentransformation $\mathcal{T} : y \in W^n \longmapsto x = Ty \in V^n$ gegeben.
					
					Dann lässt sich die lineare Abbildung in den neuen Koordinaten darstellen als
					\[
						\mathcal{G} = \mathcal{T}^{-1} \circ \mathcal{F} \circ \mathcal{T} : y \in W^n \longmapsto y' = T^{-1} ATy \in W^n.
					\]
				\end{satz}
				
				\begin{definition}
					Die $n \times n$-Matrix $B$ heisst \emph{ähnlich} zur $n \times n$-Matrix $A$, falls es eine reguläre $n \times n$-Matrix $T$ gibt, so dass
					\[
						B = T^{-1} AT .
					\]
				\end{definition}
			% paragraph: Koordinatentransformation (end)
			\paragraph{Die Norm einer Matrix} % (fold)
				\begin{definition}
					Sei $A$ eine $n \times n$-Matrix, und sei in $V^n$ eine Norm $||x||_*, x \in V^n$, gegeben. Dann ist die Zahl
					\[
						||A||_* := \sup_{x \in V^n, x \neq 0} \left\{ \frac{||Ax||_*}{||x||_*} \right\}
					\]
					\emph{Norm von A} (genauer: die \emph{von der Vektornorm $||x||_*$ induzierte Matrixnorm}).
				\end{definition}
				
				\begin{bemerkung}
					In der Definition von $||A||_*$ kann der Quotient $||Ax||_* / ||x||_*$ auch anders geschrieben werden.
					\[
						\left|\left| A \left( \frac{1}{||x||_*} x \right) \right|\right|_*.
					\]
					Für alle $x \in V^n, x \neq 0$, hat der Vektor $\frac{1}{||x||_*}x$ die Norm 1. Es gilt also
					\[
						||A||_* = \sup_{||x||_* = 1} \{||Ax||_*\}.
					\]
				\end{bemerkung}
				
				\begin{satz}
					\renewcommand{\labelenumi}{\textup{\roman{enumi})}}
					\begin{enumerate}
						\item $||A||_* \geq 0$, aus $||A||_* = 0$ folgt $A = 0$.
						\item $||\alpha A ||_* = |\alpha| ||A||_*$.
						\item $|| A + B ||_* \leq ||A||_* + ||B||_*$.
						\item $||Ax||_* \leq ||A||_* ||x||_*$.
						\item $||AB||_* \leq ||A||_* ||B||_*$.
					\end{enumerate}
				\end{satz}
			% paragraph: Die Norm einer Matrix (end)
			\paragraph{Orthogonale Abbildungen} % (fold)
				\begin{definition}
					\begin{enumerate}
						\item Die Abbildung $\mathcal{F} : x \in \Real^n \longmapsto x' = Ax \in \Real^n$ heisst \emph{orthogonal}, falls $(x', y') = (Ax, Ay) = (x, y)$ für alle $x, y \in \Real^n$ gilt.
						\item Die Abbildung $\mathcal{F} : x \in \Real^n \longmapsto x' = Ax \in \Real^n$ heisst \emph{längentreu}, falls $||x'|| = ||Ax|| = ||x||$ für alle $x \in \Real^n$ gilt.
					\end{enumerate}
				\end{definition}
				
				\begin{satz}
					Sei die lineare Abbildung $\mathcal{F} : x \in \Real^n \longmapsto x' = Ax \in \Real^n$ gegeben. Dann sind folgende Aussagen äquivalent:
					\renewcommand{\labelenumi}{\textup{\roman{enumi})}}
					\begin{enumerate}
						\item $\mathcal{F}$ ist orthogonal.
						\item $\mathcal{F}$ ist längentreu.
						\item Die Spalten von $A$ bilden eine orthonormale Basis in $\Real^n$.
						\item Die Matrix $A$ ist orthogonal, d.h. es gilt $A^T A = I$, bzw. $A^T = A^{-1}$.
					\end{enumerate}
				\end{satz}
				
				\begin{bemerkung}
					\renewcommand{\labelenumi}{\textup{\arabic{enumi})}}
					\begin{enumerate}
						\item Jede längentreue Abbildung ist \emph{winkeltreu}.
						\item Für eine orthogonal Matrix $A$ gilt $||A||_2 = 1$.
					\end{enumerate}
				\end{bemerkung}
			% paragraph: Orthogonale Abbildungen (end)
		% subsection: Lineare Selbstabbildung von Vektorräumen (end)
	% (end)
	
	\section{Das Eigenwertproblem} % (fold)
		\subsection{Eigenwerte} % (fold)
			\begin{definition}
				\renewcommand{\labelenumi}{\textup{\roman{enumi})}}
				\begin{enumerate}
					\item Eine Zahl $\lambda \in \Complex$ heisst \emph{Eigenwert der Matrix $A$}, falls es einen Vektor $x \in \Complex^n$ gibt, $x \neq 0$, so dass $Ax = \lambda x$ gilt.
					\item Ist $\lambda \in \Complex$ Eigenwert der Matrix $A$, so heisst jeder Vektor $x \in \Complex^n, x \neq 0$, für den $Ax = \lambda x$ gilt, \emph{Eigenvektor der Matrix $A$ zum Eigenwert $\lambda$}.
				\end{enumerate}
			\end{definition}
			
			\begin{satz}
				Die Zahl $\lambda \in \Complex$  ist genau dann ein Eigenwert der Matrix $A$, wenn $\det(A-\lambda I_n) = 0$ gilt.
			\end{satz}
			
			Für jede $n \times n$-Matrix $A$ ist $\det(A - \lambda I_n)$ ein Polynom $n$-ten Grades in $\lambda$.
			\begin{definition}
				Das Polynom $\det(A-\lambda I_n)$ heisst \emph{charakteristisches Polynom der Matrix $A$}. Es wird mit $P_A(\lambda)$ bezeichnet.
			\end{definition}
			
			\begin{definition}
				Ist $\lambda^*$ Eigenwert der Matrix $A$ und ist der Faktor $\lambda - \lambda^*$ genau $k$-mal in der Linearfaktorzerlegung des charakteristischen Polynoms $P_A(\lambda)$ enthalten, so heisst $k$ die \emph{algebraische Vielfachheit} von $\lambda^*$ (kürzer: $\lambda^*$ ist \emph{$k$-facher Eigenwert} von $A$).
				
				Es gelten die folgenden Aussagen:
				\begin{itemize}
					\item Jede quadratische Matrix hat mindestens einen Eigenwert.
					\item Jede $n \times n$-Matrix hat höchstens $n$ Eigenwerte.
					\item Für jeden Eigenwert ist die algebraische Vielfachheit grösser gleich $1$ und kleiner gleich $n$.
					\item Jede $n \times n$-Matrix hat genau $n$ Eigenwerte, wenn jeder Eigenwert mit seiner algebraischen Vielfachheit gezählt wird.
					\item Für jede reelle Matrix sind die Koeffizienten des charakteristischen Polynoms reell. In diesem Fall sind die Eigenwerte entweder reell oder sie treten in konjugiert komplexen Paaren auf.
					\item Für das charakteristische Polynom $P_A(\lambda) = c_n \lambda^n + c_{n-1} \lambda^{n-1} + \dots + c_1 \lambda + c_0$ der $n \times n$-Matrix $A$ gilt immer
					\[
						\begin{array}{l@{\:=\:}l}
							c_n & (-1)^n \\ [5pt]
							c_{n-1} & (-1)^{n-1} (a_{11} + a_{22} + \dots + a_{nn}) \\
							 & (-1)^{n-1} \Spur A \\ [5pt]
							c_0 & \det A .
						\end{array}
					\]
				\end{itemize}
			\end{definition}
			
			\begin{satz}
				\renewcommand{\labelenumi}{\textup{\roman{enumi})}}
				\begin{enumerate}
					\item Ähnliche Matrizen haben das gleiche charakteristische Polynom; sie haben also die gleichen Eigenwerte mit den gleichen algebraischen Vielfachheiten.
					\item Ist $B = T^{-1} AT$ und ist $x$ ein Eigenvektor von $A$ zum Eigenwert $\lambda$, so ist $y = T^{-1} x$ ein Eigenvektor von $B$ zum selbel Eigenwert $\lambda$.
				\end{enumerate}
			\end{satz}
		% subsection: Eigenwerte (end)
		\subsection{Eigenvektoren} % (fold)
			\begin{equation}
				(A-\lambda I_n)x = 0 \label{eq.eigenvek}
			\end{equation}
			
			\begin{definition}
				Ist $\lambda$ Eigenwert der Matrix $A$, dann heisst die Menge der Lösungen von \eqref{eq.eigenvek} \emph{Eigenraum von $A$ zum Eigenwert $\lambda$}. Dieser Unterraum von $\Complex^n$ wird mit $E_{\lambda}$ bezeichnet. Die Dimension des Unterraumes $E_{\lambda}$ heisst \emph{geometrische Vielfachheit} des Eigenwertes $\lambda$.
			\end{definition}
			
			\begin{satz}
				Sei $\lambda^*$ Eigenwert der Matrix $A$, dann gilt
				\begin{gather*}
					1 \leq \text{ geometrische Vielfachheit von } \lambda^* \\ \leq \text{ algebraische Vielfachheit von } \lambda^*
				\end{gather*}
			\end{satz}
			
			\begin{satz}
				Seien $\lambda_1, \dots, \lambda_k$ paarweise verschiedene Eigenwerte der Matrix $A$ und seien $u^{(1)}, \dots , u^{(k)}$ zugehörige Eigenvektoren.
				
				Dann sind die Eigenvektoren $u^{(1)}, \dots , u^{(k)}$ linear unabhängig.
			\end{satz}
			
			\begin{definition}
				Eine Basis von Eigenvektoren einer Matrix $A$ nennt man \emph{Eigenbasis zur Matrix $A$}.
			\end{definition}
			
			\begin{satz}
				Seien $g_1, g_2, \dots, g_k$ die geometrischen Vielfachheiten der paarweise verschiedenen Eigenwerte $\lambda_1, \lambda_2, \dots , \lambda_k$. In jedem Eigenraum $E_{\lambda_i}, i = 1, \dots, k$, gibt es also $g_i$ linear unabhängige Vektoren.
				
				Dann gilt: Diese $g_1 + g_2 + \dots + g_k$ Vektoren zusammen sind linear unabhängig.
			\end{satz}
			
			\begin{folgerung}
				Wenn die Summe der geometrischen Vielfachheiten einer $n \times n$-Matrix $A$ gleich $n$ ist, dann gibt es eine Eigenbasis zu $A$. Die Summe der geometrischen Vielfachheiten ist genau dann gleich $n$, wenn für jeden Eigenwert die geometrische Vielfachheit gleich der algebraischen Vielfachheit ist.
			\end{folgerung}
			
			\begin{definition}
				\renewcommand{\labelenumi}{\textup{\roman{enumi})}}
				\begin{enumerate}
					\item Eine quadratische Matrix heisst \emph{einfach}, wenn jeder Eigenwert die algebraische Vielfachheit 1 (und damit auch die geometrische Vielfachheit 1) hat.
					\item Eine quadratische Matrix heisst \emph{halbeinfach}, wenn für jeden Eigenwert die geometrische Vielfachheit gleich der algebraischen Vielfachheit ist.
				\end{enumerate}
			\end{definition}
			
			\paragraph{Diagonalisierbarkeit} % (fold)
				\begin{definition}
					Eine quadratische Matrix $A$ heisst \emph{diagonalisierbar}, falls es eine reguläre Matrix $T$ gibt, so dass die Matrix $T^{-1} AT$ eine Diagonalmatrix ist.
				\end{definition}
				
				\begin{satz}
					Für jede quadratische Matrix $A$ sind folgende Aussagen äquivalent:
					\renewcommand{\labelenumi}{\textup{\roman{enumi})}}
					\begin{enumerate}
						\item Die Matrix $A$ ist halbeinfach.
						\item Die Matrix $A$ besitzt eine Eigenbasis.
						\item Die Matrix $A$ ist diagonalisierbar.
					\end{enumerate}
				\end{satz}
				
				\begin{folgerung}
					\renewcommand{\labelenumi}{\textup{\arabic{enumi})}}
					\begin{enumerate}
						\item Bilden $u^{(1)}, u^{(2)}, \dots, u^{(n)}$ eine Eigenbasis zu $A$, dann diagonalisiert die Matrix $T = (u^{(1)} u^{(2)} \dots u^{(n)})$ die Matrix $A$, d.h. die Matrix $D := T^{-1} AT$ ist diagonal. In der Diagonalen von $D$ stehen die Eigenwerte von $A$.
						\item Umgekehrt gilt: Gibt es eine reguläre Matrix $T$ und eine Diagonalmatrix $D$, so dass $T^{-1} AT = D$ ist, dann bilden die Spalten von $T$ eine Eigenbasis zu $A$. In der Diagonalen von $D$ stehen die Eigenwerte von $A$.
					\end{enumerate}
				\end{folgerung}
			% paragraph: Diagonalisierbarkeit (end)
		% subsection: Eigenvektoren (end)
		\subsection{Das Eigenwertproblem symmetrischer Matrizen} % (fold)
			\begin{satz}
				Sei $A$ eine reelle, symmetrische Matrix. Dann gilt:
				\renewcommand{\labelenumi}{\textup{\roman{enumi})}}
				\begin{enumerate}
					\item Alle Eigenwerte von $A$ sind reell.
					\item Eigenvektoren zu verschiedenen Eigenwerten stehen senkrecht aufeinander.
				\end{enumerate}
			\end{satz}
			
			\begin{satz}
				Sei $A$ eine reelle, symmetrische Matrix. Dann gilt:
				\renewcommand{\labelenumi}{\textup{\roman{enumi})}}
				\begin{enumerate}
					\item Die Matrix $A$ ist halbeinfach (und somit diagonalisierbar).
					\item Es gibt eine orthonormale Eigenbasis zu A.
					\item Es gibt eine orthogonale Matrix $T$, so dass die Matrix $T^T AT$ diagonal ist. In der Diagonalen stehen die Eigenwerte der Matrix $A$. Die Spalten von $T$ sind die entsprechenden Eigenvektoren der Matrix $A$.
				\end{enumerate}
			\end{satz}
		% subsection: Das Eigenwertproblem symmetrischer Matrizen (end)
	% (end)
	
	% \section{Anwendungen zum Eigenwertproblem} % (fold)
	% 
	% % (end)
	% 
	% \section{Normalformen} % (fold)
	% 
	% % (end)
	% 
	% \section{Numerische Behandlung des Eigenwertproblems} % (fold)
	% 
	% % (end)
	
	\end{multicols*}
	
	\printindex
\end{document}
%!TEX root = ../Technische Dynamik Zusammenfassung.tex

\section{Variationsrechnung} % (fold)
	
	\subsection{Vorbemerkungen} % (fold)
		
		\begin{definition}
			Eine Menge $V$ heisst \emph{Vektorraum über $\R$}, wenn 
			\begin{tightitemize}
				\item für beliebige $a, b, c \in V$ gilt:
					\begin{gather*}
						a + b = b + a \tag{kommutativ} \\
						a + (b + c) = (a + b) + c \tag{assoziativ} \\
						a + 0 = 0 + a = a \tag{neutrales Element} \\
						a + (-a) = (-a) + a = 0 \tag{inverses Element}
					\end{gather*}
				\item eine Skalarmultiplikation $V \ni a \to \alpha \cdot a \in V$ mit $a \in \R$ definiert ist, dass
					\begin{gather*}
						\alpha (a + b) = \alpha a + \alpha b \\
						(\alpha + \beta) a = \alpha a + \beta a \tag{distributiv} \\
						\alpha \cdot \beta) \cdot a = \alpha \cdot (\beta \cdot a) \tag{assoziativ}
					\end{gather*}
			\end{tightitemize}
			
			Handelt es sich bei den Vektoren um \emph{Funktionen} $\vec{f}: \R^{n} \supseteq D \to \R^{m}$, dann spricht man von einem (reellen) \emph{Funktionenraum}.
		\end{definition}
		
		\begin{definition}
			Sei $V$ Vektorraum über $\R$ und $D \subseteq V$. Eine Abbildung $I: D \to \R$, die jeden Vektor $v \in D$ eine reelle Zahl $I(v)$ zuordnet, heisst \emph{Funktional} auf $D$.
		\end{definition}
		
	% subsection vorbemerkungen (end)
	
	\subsection{Die Euler-Gleichung der Variationsrechnung} % (fold)
		Finde $y(x)$ mit $y(x_0) = y_0$, $y(x_1) = y_1$ so, dass $I(y) = \int_{x_0}^{x_1} \underbrace{F(x,y(x),y'(x))}_\text{Lagrange-Funktion} \diff x \stackrel{!}{\longrightarrow}$ minimal oder stationär. \\
		Betrachte dazu \emph{beliebige} Kurvenscharen $\eta(\epsilon,x)$, wobei $\eta(\epsilon_0,x) = y(x)$ die gesuchte Kurve ist und $\eta(\epsilon \neq \epsilon_0,x) = y(x)$ Vergleichskurven sind und alle $\epsilon$ die Randbedingungen erfüllen:
		\begin{gather*}
			\eta(\epsilon,x_0) = y_0 \ \Rightarrow \ \eta_\epsilon(\epsilon,x_0) \equiv 0 \\
			\eta(\epsilon,x_1) = y_1 \ \Rightarrow \ \eta_\epsilon(\epsilon,x_1) \equiv 0
		\end{gather*}
		\emphequation{equation*}{
			h(\epsilon) := I(\eta(\epsilon,\cdot)) = \int_{x_0}^{x_1} F(x,
				\underbrace{\eta(\epsilon,x)}_{y},
				\underbrace{\eta_x(\epsilon,x)}_{y'}
			) \diff x \stackrel{!}{\longrightarrow} \text{stationär}
		}
		Ist $y(x) = \eta(\epsilon_0, x)$ eine Lösung des Variationsproblems, dann muss $h'(\epsilon_0) \stackrel{!}{=} 0$ sein.

		\paragraph{Fundamental-Lemma der Variationsrechnung} % (fold)
			Gilt für eine stetige Funktion $F: [a, b] \to \R$, dass $\int_a^b f(x) \cdot g(x) \diff x = 0$ für jede glatte Funktion $g(x)$ mit $g(a) = g(b) = 0$, dann ist $f \equiv 0$.
		% paragraph Fundamental-Lemma der Variationsrechnung (end)
		
		\paragraph{Euler-Gleichung} % (fold)
			ist eine notwendige Bedingung für ein Minimum und ist eine gewöhnliche Dgl. 2. Ordnung.
			\emphequation{equation*}{
				\frac{\partial F}{\partial y} - \frac{\diff}{\diff x}\parens{\frac{\partial F}{\partial y'}} = 0
			}
			\subparagraph{Sonderfälle} % (fold)
				\begin{enumerate}[a)]
					\item $\boldsymbol{F = F(x,y')}$
						$
							\Rightarrow \ F_{y'}(x,y'(x)) = \const
						$ \\
						Löse nach $y'$ auf; integriere $\Rightarrow \ y(x)$.
						
					\item $\boldsymbol{F = F(x,y)}$
						$
							\Rightarrow \ F_y(x,y(x)) = 0
						$
						
					\item $\boldsymbol{F = F(y,y')}$
						$
							\Rightarrow \ F - F_{y'} \cdot y' = \const
						$ \\
						Löse nach $y'(x) = \Phi(y(x))$ auf; integriere $\Rightarrow \ y(x)$.
				\end{enumerate}
			% subparagraph Sonderfälle (end)
		% paragraph Euler-Gleichung (end)
	% subsection (end)
	
	\subsection{Variationsprobleme und gewöhnliche Differentialgleichungen} % (fold)
		\begin{description}
			\item[Gegeben] sei eine Dgl der Form
				\begin{equation}\label{eq:selbstadj}
					a_2(x) y'' + a_1(x) y' + a_0(x) y = f(x) \ .
				\end{equation}
				
			\item[Gesucht] ist nun das zugehörige Variationsproblem.
		
			\item[Schritt 1:] Stelle ``selbstadjungierte Form'' von \eqref{eq:selbstadj} her:
				\[
				 p(x) y'' + \underbrace{p(x) \frac{a_1(x)}{a_2(x)}}_{=: p'(x)} y' + \underbrace{p(x) \frac{a_0(x)}{a_2(x)}}_{=: q(x)} y = \underbrace{p(x) \frac{f(x)}{a_2(x)}}_{=:h(x)}
				\]
				\begin{gather*}
					\Rightarrow p(x) = e^{\frac{a_1}{a_2} \cdot x} \\
					(p(x) y')' - (h(x) - q(x) y) = 0 = \frac{\diff}{\diff x}(F_{y'}) - F_y
				\end{gather*}
			
			\item[Schritt 2:] Finde Lagrange-Funktion:
				\begin{align*}
					F_{y'} &= p(x) y' \\
					\Rightarrow F(x,y,y') &= \half p(x) y'^2 - \half q(x) y^2 + h(x) y
				\end{align*}
				
		\end{description}
	% subsection Variationsprobleme und gewöhnliche Differentialgleichungen (end)
	
	\subsection{Randbedingungen} % (fold)
		\begin{description}
			\item[Natürliche Randbedingungen:]
				Gesucht $y(x)$ mit $y(x_0) = y_0$ und $y(x_1)$ frei, so dass 
				\[
					I(y) = \int_{x_0}^{x_1} F(x,y,y') \diff x \longrightarrow \text{stationär.}
				\]
			
				\begin{enumerate}[(i)]
					\item Betrachte zuerst $\eta$, für die $\eta_\epsilon(\epsilon_0, x_1) = 0$
						\[
							\Rightarrow F_y - \frac{\diff}{\diff x} F_{y'} = 0
						\]
					\item Betrachte jetzt alle anderen
						\[
							\Rightarrow F_{y'}(x_1, y(x_1), y'(x_1)) = 0
						\]
				\end{enumerate}
			
			\item[Transversalitätsbedingungen:]
				Gesucht ist $y(x)$ mit $y(x_0) = y_0$ und $y(b) = f(b)$ mit gegebenem $f(x)$ und noch \emph{unbekanntem} $b$ so, dass
				\[
					I(y) = \int_{x_0}^b F(x,y,y') \diff x \longrightarrow \text{stationär.}
				\]
				
				$\eta(\epsilon,b(\epsilon)) = f(b(\epsilon))$ nach $\epsilon$ abgeleitet und nach $b_\epsilon$ aufgelöst, ergibt
				\[
					b_\epsilon(\epsilon) = \frac{\eta_\epsilon(\epsilon,b(\epsilon))}{f'(b(\epsilon)) - \eta_x(\epsilon,b(\epsilon))} 
				\]
				
				\textbf{Resultat:}
				\begin{align*}
					\text{(i)}\quad &F_y - \Diff{}{x} (F_{y'}) = 0 \\
					\text{(ii)}\quad &F_{y'}(b, y(b), y'(b)) + \frac{F((b),y(b),y'(b))}{f'(b) - y'(b)} = 0
				\end{align*}
			
			\item[Modifizierte Randbedingungen:]
				Gesucht $y(x)$ mit $y(x_0) = y_0$ und $y(x_1)$ frei so, dass
				\[
					I(y) = \int_{x_0}^{x_1} F(x,y,y') \diff x - H(y(x_1)) \longrightarrow \text{stationär}
				\]
				für gewisse gegebene Funktion $H(y)$.
				
				\textbf{Stationaritätsbedingung:}
				\begin{align*}
					\text{(i)}\quad &F_y - \Diff{}{x} (F_{y'}) = 0 \\
					\text{(ii)}\quad &F_{y'}(x, y(x_1), y'(x_1)) = H'(y(x_1))
				\end{align*}
		\end{description}
	% subsection Randbedingungen (end)
	
	\subsection{Variation mit höheren Ableitungen} % (fold)
		Gesucht ist $y(x)$ so, dass
		\[
			I(y) = \int_{x_0}^{x_1} F \left( x, y(x), y'(x),y''(x), \ldots, y^{(n)}(x) \right) \diff x \rightarrow \text{ stat.}
		\]
		unter den $2n$ Randbedingungen
		\begin{align*}
		 y(x_0) &= y_0 & y'(x_0) &= y_0^1, &\ldots,& &y^{(n-1)}(x_0) &= y_0^{n-1} \\
		 y(x_1) &= y_1 & y'(x_1) &= y_1^1, &\ldots,& &y^{(n-1)}(x_1) &= y_1^{n-1} 
		\end{align*}
		Euler-Gleichung für Gewöhnliche Differentialgleichungen $2n$-ter Ordnung (braucht $2n$ Randbedingungen):
		\emphequation{equation*}{
			F_y - \Diff{}{x} F_{y'} + \Diff{}{x^2} F_{y''} - \Diff{}{x^3} F_{y'''} + \ldots + (-1)^{n} \Diff{}{x^{n}} F_{y^n} = 0
		}
	% subsection Variation mit höheren Ableitungen (end)
	
	\subsection{Variation mit Funktionen mehrerer Variablen} % (fold)
		Gesucht ist $x(x,y,z)$ so, dass
		\[
			I(w) = \iiint_G F(x,y,z,w,w_x,w_y,w_z) \diff V \longrightarrow \text{stationär}
		\]
		für gegebenes $w(x,y,z) = w_0(x,y,z)$ auf $\delta G$.
		
		Vergleichsfelder $\eta(\epsilon,x,y,z)$ so, dass für $(x,y,z) \in \delta G$
		\begin{align*}
			\eta(\epsilon,x,y,z) &= w_0(x,y,z) \\
			\Rightarrow \eta_\epsilon(\epsilon,x,y,z) &= 0
		\end{align*}
		
		Euler-Gleichung für mehrere Variablen:
		\emphequation{equation*}{
			F_w - \frac{\partial}{\partial x} F_{w_x} - \frac{\partial}{\partial y} F_{w_y} - \frac{\partial}{\partial z} F_{w_z} = 0
		}
	% subsection Variation mit Funktionen mehrerer Variablen (end)
	
	\subsection{Variation mit mehreren Funktionen} % (fold)
		Gesucht ist $\vec q(t) = (q_1(t),\dots,q_n(t))^\transp$ so, dass
		\[
			I(\vec q) = \int_{t_0}^{t_1} L(t,\vec q, \dot{\vec q}) \diff t \longrightarrow \text{stationär}
		\]
		mit $\vec q(t_0) = \vec q_0$, $\vec q(t_1) = \vec q_1$.
		
		Vergleichskurven $\vec \eta(\epsilon,t)$ so, dass $\vec \eta(\epsilon_0, t) = \vec q(t)$.
		
		Euler-Gleichung:
		\emphequation{equation*}{
			\frac{\partial L}{\partial \vec q} - \frac{\diff}{\diff t}\parens{\frac{\partial L}{\partial \dot{\vec q}}} = 0 
		}
		
		Sonderfall c) ($L$ hängt nicht von $t$ ab) ist wichtig für die Dynamik:
		\emphequation{equation*}{
			L - \parens{\frac{\partial L}{\partial \vec q} } \dot{\vec q} =: - H_0 = \const
		}
	% subsection Variation mit mehreren Funktionen (end)
% section variationsrechnung (end)
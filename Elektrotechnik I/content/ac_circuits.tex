%!TEX root = ../Elektrotechnik I.tex

\subsection{Superposition Analysis} % (fold)
	The current through, or voltage across an elements in a linear network is equal to the algebraic sum of the currents or voltages produced by each independant source.
	
	\begin{align*}
		V &= V_{m1} \cdot \cos(\omega t + \theta_1) + V_{m2} \cdot \cos(\omega t + \theta_2) \\
		&= V_{m3} \cdot \cos(\omega t + \theta_3)
	\end{align*}
	where
	\begin{align*}
		V_{m3} &= \sqrt{V_{m1}^2 + V_{m2}^2 + 2 \cdot V_{m1} \cdot V_{m2} \cdot \cos(\theta_1 - \theta_2)} \\
		\theta_3 &= \arctan \left(
			\frac{
				V_{m1} \cdot \sin \theta_1 + V_{m2} \cdot \sin \theta_2
			}{
				V_{m1} \cdot \cos \theta_1 + V_{m2} \cdot \cos \theta_2
			}
		\right)
	\end{align*}
		
		\paragraph{With Phasors:} % (fold)
			Generally easier.
			
			\begin{gather*}
				S = V_m \angle \theta = V_m (\cos \theta + \iu \sin \theta) = V_m \cdot \eu^{\iu\theta} \\
				|S| = \sqrt{\Re^2 + \Im^2} \qquad \angle S = \arctan \frac{\Im}{\Re}
			\end{gather*}
			
			\begin{tabular}{lll}
				\textbf{Resistor:} & $\theta_R = 0$ & $\frac{V}{I} = R$ \\ [2ex]
				\textbf{Capacitor:} & $\theta_C = -90\degree$ & $\frac{V}{I} = \frac{1}{\iu\omega C} = -\iu \frac{1}{\omega C}$ \\ [2ex]
				\textbf{Inductor:} & $\theta_L$ & $\frac{V}{I} = \iu\omega L$ \\
			\end{tabular}
		% paragraph: With Phasors: (end)
% subsection: Superposition (end)

\subsection{Impedance} % (fold)
	\[
		Z = \frac{V}{I} \qquad Z = R + \iu X
	\]
	
	\[
		\Im(Z) \left\{\begin{array}{@{}l@{\quad}l}
			>0 & \text{inductive} \\
			= 0 & \text{resistive} \\
			<0 & \text{capacitive}
		\end{array}\right.
	\]
	
	Current/voltage divider rule and connections in series/parallel are analogous to resistance.
	
	\paragraph{Admittance:} % (fold)
		$
			Y = \frac{1}{Z} \sunit{Siemens}
		$
	% subsubsection: Admittance (end)
	
	\paragraph{Voltage drop across impedance:} % (fold)
		$V$ is the voltage drop, $I$ the current and $Z$ the impedance consisting of a resistor and an inductor in series.
		
		\begin{wrapfigure}[0]{r}{.35\columnwidth}
			\vspace{-.5cm}
			\circuitw{voltage_across_impedance}{.35\columnwidth}
		\end{wrapfigure}
		
		\begin{align*}
			V &= \abs{Z} \cdot I \\
			V_R &= R \cdot I = V \cdot \cos \theta \\
			V_L &= \omega L \cdot I = V \cdot \sin \theta
		\end{align*}
	% paragraph: Voltage drop across impedance: (end)
% subsection: Impedance (end)

\subsection{Frequency Response} % (fold)
	Describes the behaviour of an AC circuit as a function of the signal frequency. Consists of two parts which are plotted in a Bode diagram:
	
	\begin{description}
		\item[Transfer Function:] $\displaystyle
			H(\iu \omega) = \frac{V_{\text{out}}}{V_{\text{in}}}
		$
		\item[Amplitude Response:] $\abs{H(\iu \omega)} \quad$
		in dB: $\quad 20\cdot \log_{10} \abs{H(\iu \omega)}$
		\item[Phase Response:] $\angle H(\iu \omega)$
	\end{description}
% subsection: Frequency Response (end)
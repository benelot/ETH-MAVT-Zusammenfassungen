%!TEX root = ../Elektrotechnik I.tex

\subsection{Superposition} % (fold)
	
	In a network with multiple sources
	\begin{tightitemize}
		\item the potential in each node
		\item the current in each branch
	\end{tightitemize}
	equals the sum of the individual contributions of all sources.
	
	To calculate the contribution of each source, all other sources have to be replaced as follows:
	\begin{tightitemize}
		\item current sources become open circuits
		\item voltage sources become short circuits
	\end{tightitemize}
	
% subsection: Superposition (end)

\subsection{Network Equivalents} % (fold)
	
	\resizebox{\columnwidth}{!}{
	\circuit{thevenin_simple}
	\circuit{norton_simple}
	}
	
	\begin{enumerate}
		\item remove the external load $R_L$
		\item mark the terminals with $a$ and $b$ (or $+/-$)
		\item 
			\begin{description}
				\item[Thévenin:] calculate $E_{Th}$ by finding the open-circuit voltage between $a$ and $b$
				\item[Norton:] calculate $I_N$ by finding the short-circuit current between $a$ and $b$
			\end{description}
		\item calculate $R_{Th} \equiv R_N$ by replacing voltage sources with short circuits and current sources with open circuits
	\end{enumerate}
	
	\paragraph{Conversion Between Norton and Thévenin:} % (fold)
		\begin{gather*}
			E_{Th} = I_N \cdot R_N \qquad
			I_N = \frac{E_{Th}}{R_N} \qquad R_N = R_{Th}
		\end{gather*}
	% paragraph: Conversion Between Norton and Thévenin (end)
	
	\paragraph{Maximum Power Transfer:} % (fold)
		$R_L$ will receive maximum power when it equals $R_{Th}$ or $R_N$.

		\[
			P_{L,max} = \frac{E_{Th}^2}{4 R_{Th}} = \frac{I_N^2 \, R_N}{4}
		\]
	% paragraph: Maximum Power Transfer: (end)
	
% subsection: Network Equivalents (end)
%!TEX root = ../Elektrotechnik I.tex

\subsection{I-V Relationship for Circuit Elements} % (fold)
	\paragraph{Resistor:} % (fold)
		\begin{align*}
			v_R &= E_m \sin (\omega t + \theta) \\
			i_R &= \frac{E_m}{R} \sin (\omega t + \theta)
		\end{align*}
		Both have same phase.
	% paragraph: Resistor: (end)
	\paragraph{Capacitor:} % (fold)
		\begin{align*}
			v_C &= E_m \sin(\omega t + \theta) \\
			i_C &= C \cdot \Diff{v_C}{t} \\
			&= \omega\:C\:E_m\:\cos (\omega t + \theta) \\
			&= \omega\:C\:E_m\:\sin (\omega t + \theta + 90\degree)
		\end{align*}
		Voltage is \SI{90}{\degree} behind current.
	% paragraph: Capacitor: (end)
	\paragraph{Inductor:} % (fold)
		\begin{align*}
			v_L &= L \cdot \Diff{i_L}{t} \\
			i_L &= -\frac{E_m}{\omega L} \cdot \cos(\omega t + \theta) \\
			&= \frac{E_m}{\omega L} \cdot \sin(\omega t + \theta - 90\degree)
		\end{align*}
		Current is \SI{90}{\degree} behind voltage.
	% paragraph: Inductor: (end)
% subsection: I-V Relationship for Circuit Elements (end)